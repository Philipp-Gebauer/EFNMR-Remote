% !TEX root = main.tex
\section{Fehlerdiskussion und Fazit}
In Bezug auf die Beobachtung der Sonne kann ein durchweg positives Fazit gezogen werden.
Einige Annahmen -- die Sonne als Punktquelle oder das Teleskop als Lochblende aufzufassen -- und daran anknüpfende Folgerungen, wie das Erhalten der $\sinc$-Funktion oder das Berechnen des Auflösungsvermögens mittels FWHM wurden durch die guten Ergebnisse und anschaulichen Grafiken legitimiert.
Bis auf die Positioniergenauigkeit -- $\SI{0.976 \pm 0.034}{\degree}$ gegenüber $\SI{0.5}{\degree}$ \cite{Usermanual} -- und den beiden Durchmessern -- $D_{\text{az}} = \SI{1.895 \pm 0.032}{\metre}$ und $D_{\text{alt}} = \SI{2.625 \pm 0.056}{\metre}$ gegnüber $D = \SI{2.3}{\metre}$ \cite{Usermanual} -- des Teleskops liegen alle ermittelten Charakteristika auf Grundlage der Unsicherheiten in guter Übereinstimmung mit den Werten der Projekt-Dokumentation \cite{Usermanual}.
Auch die Positioniergenauigkeit liegt in derselben Größenordnung und widerspricht somit nicht gänzlich der Erwartung.
Das Auflösungsvermögen ($\SI{6.6 \pm 1.1}{\degree}$ zu $\SI{6}{\degree}$ \cite{Usermanual}) und das Rückrechnen auf den Mittelwert des Teleskopdurchmessers ($\SI{2.23 \pm 0.37}{\metre}$ zu $\SI{2.3}{\metre}$ \cite{Usermanual}) zeigen wie erwähnt gute Übereinstimmungen.

Die Unsicherheiten der betrachteten Werte liegen allesamt in sinnvollen Größenordnungen und lassen darauf schließen, dass kein systematischer Fehler während der Durchführung und Auswertung des Versuchs auftrat und auch mögliche Defekte oder äußere Einflüsse keine zu große Beeinträchtigung der Messung darstellten.
Auffällig war, dass stets bei der Variation des Höhenversatzwinkels schmalere $\sinc$-Profile respektive \textSC{Gauss}-Kurven auftraten.
Hierfür wurden verschiedene Ansätze diskutiert.
Zum einen wurden Witterung und äußere Verhältnisse am Versuchstag in Erwägung gezogen.
Da das Teleskop allerdings in verschiedenen Ausrichtungen Messungen vornahm, könnte ein stets aus einer Richtung wehender Wind zwar Einfluss auf die Positioniergenauigkeit haben, aber keine solche kontinuierlich auftretende Abweichung erklären.
Eventuell vorliegende Defekte des Parabolspiegels, welche den effektiven Beitrag bestimmter Teleskopareale zur Messung schmälern, sowie daraus resultierende Abweichungen der Justierung oder Schäden der Mechanik des Teleskops sind hierbei deutlich plausiblere Fehlerquellen.
Mögliche Verbesserungen hinsichtlich der Messgenauigkeit könnten durch eine größere Zahl Messungen oder beispielsweise kleinschrittigere Variation der Versatzwinkel in Azimut und Höhenkomponente erreicht werden.
Die Defekte am Teleskop könnten durch regelmäßigere Wartungen minimiert werden.
Zudem könnten diese bei der Justierung berücksichtigt werden, um die Messergebnisse zu verbessern. \newpage

Auch die Messungen und Berechnungen der Milchstraße lieferten sehr gute Werte, welche stets in guter Übereinstimmung mit der Literatur sind. 
Ausschlaggebende erfolgsfaktoren dieses Versuchsteils sind zum einen die nahezu konstante Geschwindigkeit der Körper in der Milchstraße von $\SI{210.9 \pm 3.1}{\frac{km}{s}}$ (Literaturwert $\SI{220}{\frac{km}{s}}$ \cite{LSR}), was deutlich auf eine unbekannte Energie und Materie im Universum hindeutet, zum anderen die sehr präzise Kartografie der Milchstraße.
Hier sind die Seitenarme \textit{Cygnus}, \textit{Perseus} und \textit{Orion} anhand eines Vergleichs mit der Literatur bestimmbar.
Somit war ein hinreichend präzises Vermessen der Milchstraße gewährleistet. 
Wenn allerdings noch bessere Ergebnisse erzielt werden sollen, so müssen einige Unsicherheitsquellen und deren Folgen in Betracht gezogen werden. 
Eine Unsicherheitsquelle, welche Auswirkungen auf den gesamten Versuchsteil hat, ist die Integrationszeit der vermessenen Spektren. 
Denn wenn das Spektrum stärkeres Rauschen aufweist, so liefern die \textsc{Gauss}-Funktionen eine größere Unsicherheit und die Maxima sind schlechter auswertbar. 
Wie bereits diskutiert, verbessert sich das Rauschen des Spektrums bei längeren Integrationszeiten. 
Wenn also präzisere Messwerte erreicht werden wollen, so muss die Integrationszeit erhöht werden. 
Des Weiteren kann die konstante Geschwindigkeit der Wasserstoffwolken durch eine Ergänzung von zusätzlichen Messwerten genauer analysiert werden. 
Denn je mehr Messwerte vorhanden sind, desto genauer lässt sich ein Mittelwert ermitteln. 
Viele Messwerte sind auch für die Katografie der Milchstraße von Vorteil.
Denn somit fallen Fehlerpunkte nicht mehr so stark ins Gewicht und die Galaxiearme sind noch besser identifizierbar. \\

Trotz aller diskutierten Unsicherheiten ist dieser Versuch mit den verwendeten Parametern eine sehr gute Möglichkeit, gute Ergebnisse zu erzielen.
Wenn allerdings noch genauere Ergebnisse erreicht werden sollen, so müssen die genannten Verbesserungen bezüglich der Unsicherheiten in Betracht gezogen werden.