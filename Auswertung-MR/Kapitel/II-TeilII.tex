% !TEX root = main.tex
\section{Teil II -- Anwendung}
\label{sec:Teil2}
Der zweite Teil des Berichts legt den Fokus auf mögliche Anwendungen von EFNMR.
Dabei werden die gewonnen Messwerte des zweiten Versuchstags aufbereitet und analysiert. Inhaltlich wird der Einfluss der Konzentration von Kontrastmitteln wie Kupfer oder Mangan auf die Relaxationszeiten $T_1$ und $T_2$ und sich dadurch ergebende Konsequenzen, bildgebende Verfahren in 1D und 2D, die Spin-Spin Kopplung als Instrument zur chemischen Strukturanalse sowie die Selbstdiffusion von Wassermolekülen diskutiert.




% Im zweiten Teil diese Berichts soll sich nun auf Anwendungsmöglichkeiten der Magnetresonanz Technologie im Erdmagnetfeld konzentriert werden. 

% Die Grundlagen zu den NMR Spulen, sowie der Signalverarbeitung wurden hierbei bereits in den Abschnitten \ref{sec:Introduction} und \ref{sec:Aufbau} besprochen. In Kapitel \ref{sec:CoilAnalyssis} wird beschrieben, dass die B$_1$-Spule als LCR-Schwingkreis aufgeffasst werden kann und die Resonanzbedingung durch Gleichung \eqref{eq: larmorcalc} gegeben ist.