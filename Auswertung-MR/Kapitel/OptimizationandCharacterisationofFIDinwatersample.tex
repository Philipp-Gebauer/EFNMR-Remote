% !TEX root = main.tex
\section{Optimization and Characterisation of FID in water sample}
\label{sec:OptimizationandCharacterisationofFIDinwatersample}

\textcolor{red}{insert previous values}

ask: is it ok to explain what autoshim does and don´t plot any graphs at all? We don´t have data for an example plot

\begin{figure}[H]
    \centering
    \input{plots/B1dauer.tex}
    \caption[]{ask: periodicity due to duration of B$_1$, \SI{0}{\degree} -> \SI{90}{\degree} -> \SI{180}{\degree} -> \SI{270}{\degree}?}
    \label{fig:B1dauer}
\end{figure}

\begin{figure}[H]
    \centering
    \input{plots/138_puls_and_collect_number_of_delay25ms_Pulseduration0_27ms.tex}
    \caption[]{}
    \label{fig:pulsedurationbeispiel}
\end{figure}

\begin{figure}[H]
    \centering
    \input{plots/Pulsandcollect138.tex}
    \caption[]{ask: what is the peak corresponding? hydrogen signal?}
    \label{fig:Pulsandcollect}
\end{figure}



\begin{figure}[H]
    \centering
    % GNUPLOT: LaTeX picture with Postscript
\begingroup
  % Encoding inside the plot.  In the header of your document, this encoding
  % should to defined, e.g., by using
  % \usepackage[cp1252,<other encodings>]{inputenc}
  \inputencoding{cp1252}%
  \makeatletter
  \providecommand\color[2][]{%
    \GenericError{(gnuplot) \space\space\space\@spaces}{%
      Package color not loaded in conjunction with
      terminal option `colourtext'%
    }{See the gnuplot documentation for explanation.%
    }{Either use 'blacktext' in gnuplot or load the package
      color.sty in LaTeX.}%
    \renewcommand\color[2][]{}%
  }%
  \providecommand\includegraphics[2][]{%
    \GenericError{(gnuplot) \space\space\space\@spaces}{%
      Package graphicx or graphics not loaded%
    }{See the gnuplot documentation for explanation.%
    }{The gnuplot epslatex terminal needs graphicx.sty or graphics.sty.}%
    \renewcommand\includegraphics[2][]{}%
  }%
  \providecommand\rotatebox[2]{#2}%
  \@ifundefined{ifGPcolor}{%
    \newif\ifGPcolor
    \GPcolorfalse
  }{}%
  \@ifundefined{ifGPblacktext}{%
    \newif\ifGPblacktext
    \GPblacktexttrue
  }{}%
  % define a \g@addto@macro without @ in the name:
  \let\gplgaddtomacro\g@addto@macro
  % define empty templates for all commands taking text:
  \gdef\gplbacktext{}%
  \gdef\gplfronttext{}%
  \makeatother
  \ifGPblacktext
    % no textcolor at all
    \def\colorrgb#1{}%
    \def\colorgray#1{}%
  \else
    % gray or color?
    \ifGPcolor
      \def\colorrgb#1{\color[rgb]{#1}}%
      \def\colorgray#1{\color[gray]{#1}}%
      \expandafter\def\csname LTw\endcsname{\color{white}}%
      \expandafter\def\csname LTb\endcsname{\color{black}}%
      \expandafter\def\csname LTa\endcsname{\color{black}}%
      \expandafter\def\csname LT0\endcsname{\color[rgb]{1,0,0}}%
      \expandafter\def\csname LT1\endcsname{\color[rgb]{0,1,0}}%
      \expandafter\def\csname LT2\endcsname{\color[rgb]{0,0,1}}%
      \expandafter\def\csname LT3\endcsname{\color[rgb]{1,0,1}}%
      \expandafter\def\csname LT4\endcsname{\color[rgb]{0,1,1}}%
      \expandafter\def\csname LT5\endcsname{\color[rgb]{1,1,0}}%
      \expandafter\def\csname LT6\endcsname{\color[rgb]{0,0,0}}%
      \expandafter\def\csname LT7\endcsname{\color[rgb]{1,0.3,0}}%
      \expandafter\def\csname LT8\endcsname{\color[rgb]{0.5,0.5,0.5}}%
    \else
      % gray
      \def\colorrgb#1{\color{black}}%
      \def\colorgray#1{\color[gray]{#1}}%
      \expandafter\def\csname LTw\endcsname{\color{white}}%
      \expandafter\def\csname LTb\endcsname{\color{black}}%
      \expandafter\def\csname LTa\endcsname{\color{black}}%
      \expandafter\def\csname LT0\endcsname{\color{black}}%
      \expandafter\def\csname LT1\endcsname{\color{black}}%
      \expandafter\def\csname LT2\endcsname{\color{black}}%
      \expandafter\def\csname LT3\endcsname{\color{black}}%
      \expandafter\def\csname LT4\endcsname{\color{black}}%
      \expandafter\def\csname LT5\endcsname{\color{black}}%
      \expandafter\def\csname LT6\endcsname{\color{black}}%
      \expandafter\def\csname LT7\endcsname{\color{black}}%
      \expandafter\def\csname LT8\endcsname{\color{black}}%
    \fi
  \fi
    \setlength{\unitlength}{0.0500bp}%
    \ifx\gptboxheight\undefined%
      \newlength{\gptboxheight}%
      \newlength{\gptboxwidth}%
      \newsavebox{\gptboxtext}%
    \fi%
    \setlength{\fboxrule}{0.5pt}%
    \setlength{\fboxsep}{1pt}%
\begin{picture}(7200.00,5040.00)%
    \gplgaddtomacro\gplbacktext{%
      \csname LTb\endcsname%%
      \put(682,704){\makebox(0,0)[r]{\strut{}$0$}}%
      \put(682,1161){\makebox(0,0)[r]{\strut{}$10$}}%
      \put(682,1618){\makebox(0,0)[r]{\strut{}$20$}}%
      \put(682,2076){\makebox(0,0)[r]{\strut{}$30$}}%
      \put(682,2533){\makebox(0,0)[r]{\strut{}$40$}}%
      \put(682,2990){\makebox(0,0)[r]{\strut{}$50$}}%
      \put(682,3447){\makebox(0,0)[r]{\strut{}$60$}}%
      \put(682,3905){\makebox(0,0)[r]{\strut{}$70$}}%
      \put(682,4362){\makebox(0,0)[r]{\strut{}$80$}}%
      \put(682,4819){\makebox(0,0)[r]{\strut{}$90$}}%
      \put(814,484){\makebox(0,0){\strut{}$1830$}}%
      \put(1613,484){\makebox(0,0){\strut{}$1832$}}%
      \put(2411,484){\makebox(0,0){\strut{}$1834$}}%
      \put(3210,484){\makebox(0,0){\strut{}$1836$}}%
      \put(4008,484){\makebox(0,0){\strut{}$1838$}}%
      \put(4807,484){\makebox(0,0){\strut{}$1840$}}%
      \put(5605,484){\makebox(0,0){\strut{}$1842$}}%
      \put(6404,484){\makebox(0,0){\strut{}$1844$}}%
      \put(4208,2304){\makebox(0,0)[l]{\strut{}FWHM $= \SI{1.177 \pm 0.042}{\hertz}$}}%
      \put(4208,2762){\makebox(0,0)[l]{\strut{}$\sigma =  \SI{0.500 \pm 0.018}{\hertz}$}}%
    }%
    \gplgaddtomacro\gplfronttext{%
      \csname LTb\endcsname%%
      \put(209,2761){\rotatebox{-270}{\makebox(0,0){\strut{}Amplitude in $\si{\mu \volt}$}}}%
      \put(3808,154){\makebox(0,0){\strut{}Frequency in $\si{\hertz}$}}%
      \csname LTb\endcsname%%
      \put(5816,4646){\makebox(0,0)[r]{\strut{}magnitude spectrum}}%
      \csname LTb\endcsname%%
      \put(5816,4426){\makebox(0,0)[r]{\strut{}voigt-profile}}%
      \csname LTb\endcsname%%
      \put(5816,4206){\makebox(0,0)[r]{\strut{}\textsc{Gauss}-Fit}}%
    }%
    \gplbacktext
    \put(0,0){\includegraphics{plots/Pulsandcollect138_delay_25_gauss}}%
    \gplfronttext
  \end{picture}%
\endgroup

    \caption[]{ask: gauss or voigt. this is gauss\newline
    longer acquisition \SI{25}{\milli \second} -> only hydrogen siganal? is the peak the same than in the previous diagramm? \newline
    integral under curve with our measured fit?\newline
    signal to noise ratio: what to do?
    \\
    calculate: amplitude}
    \label{fig:Pulsandcollect138_delay_25_gauss}
\end{figure}
\begin{figure}[H]
    \centering
    \input{plots/Pulsandcollect138_delay_25_voigt.tex}
    \caption[]{ask: gauss or voigt. this is voigt}
    \label{fig:PulsandcollePulsandcollect138_delay_25_voigtctlangerdelay}
\end{figure}

\textcolor{red}{real and imaginary signal}