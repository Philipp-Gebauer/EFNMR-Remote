% !TEX root = main.tex
\section{T1 und T2 Relaxationszeit für Wasser und mit Zusatzmitteln}
\ce{Cu^2+} \ce{Mn^2+}

\begin{figure}[H]
    \centering
    \input{plots/T1Wasser.tex}
    \caption{T1 Messung von Wasser}
\end{figure}

\begin{figure}[H]
    \centering
    \input{plots/T2Wasser.tex}
    \caption{T2 Messung von Wasser}
\end{figure}

\begin{figure}[H]
    \centering
    % GNUPLOT: LaTeX picture with Postscript
\begingroup
  % Encoding inside the plot.  In the header of your document, this encoding
  % should to defined, e.g., by using
  % \usepackage[cp1252,<other encodings>]{inputenc}
  \inputencoding{cp1252}%
  \makeatletter
  \providecommand\color[2][]{%
    \GenericError{(gnuplot) \space\space\space\@spaces}{%
      Package color not loaded in conjunction with
      terminal option `colourtext'%
    }{See the gnuplot documentation for explanation.%
    }{Either use 'blacktext' in gnuplot or load the package
      color.sty in LaTeX.}%
    \renewcommand\color[2][]{}%
  }%
  \providecommand\includegraphics[2][]{%
    \GenericError{(gnuplot) \space\space\space\@spaces}{%
      Package graphicx or graphics not loaded%
    }{See the gnuplot documentation for explanation.%
    }{The gnuplot epslatex terminal needs graphicx.sty or graphics.sty.}%
    \renewcommand\includegraphics[2][]{}%
  }%
  \providecommand\rotatebox[2]{#2}%
  \@ifundefined{ifGPcolor}{%
    \newif\ifGPcolor
    \GPcolorfalse
  }{}%
  \@ifundefined{ifGPblacktext}{%
    \newif\ifGPblacktext
    \GPblacktexttrue
  }{}%
  % define a \g@addto@macro without @ in the name:
  \let\gplgaddtomacro\g@addto@macro
  % define empty templates for all commands taking text:
  \gdef\gplbacktext{}%
  \gdef\gplfronttext{}%
  \makeatother
  \ifGPblacktext
    % no textcolor at all
    \def\colorrgb#1{}%
    \def\colorgray#1{}%
  \else
    % gray or color?
    \ifGPcolor
      \def\colorrgb#1{\color[rgb]{#1}}%
      \def\colorgray#1{\color[gray]{#1}}%
      \expandafter\def\csname LTw\endcsname{\color{white}}%
      \expandafter\def\csname LTb\endcsname{\color{black}}%
      \expandafter\def\csname LTa\endcsname{\color{black}}%
      \expandafter\def\csname LT0\endcsname{\color[rgb]{1,0,0}}%
      \expandafter\def\csname LT1\endcsname{\color[rgb]{0,1,0}}%
      \expandafter\def\csname LT2\endcsname{\color[rgb]{0,0,1}}%
      \expandafter\def\csname LT3\endcsname{\color[rgb]{1,0,1}}%
      \expandafter\def\csname LT4\endcsname{\color[rgb]{0,1,1}}%
      \expandafter\def\csname LT5\endcsname{\color[rgb]{1,1,0}}%
      \expandafter\def\csname LT6\endcsname{\color[rgb]{0,0,0}}%
      \expandafter\def\csname LT7\endcsname{\color[rgb]{1,0.3,0}}%
      \expandafter\def\csname LT8\endcsname{\color[rgb]{0.5,0.5,0.5}}%
    \else
      % gray
      \def\colorrgb#1{\color{black}}%
      \def\colorgray#1{\color[gray]{#1}}%
      \expandafter\def\csname LTw\endcsname{\color{white}}%
      \expandafter\def\csname LTb\endcsname{\color{black}}%
      \expandafter\def\csname LTa\endcsname{\color{black}}%
      \expandafter\def\csname LT0\endcsname{\color{black}}%
      \expandafter\def\csname LT1\endcsname{\color{black}}%
      \expandafter\def\csname LT2\endcsname{\color{black}}%
      \expandafter\def\csname LT3\endcsname{\color{black}}%
      \expandafter\def\csname LT4\endcsname{\color{black}}%
      \expandafter\def\csname LT5\endcsname{\color{black}}%
      \expandafter\def\csname LT6\endcsname{\color{black}}%
      \expandafter\def\csname LT7\endcsname{\color{black}}%
      \expandafter\def\csname LT8\endcsname{\color{black}}%
    \fi
  \fi
    \setlength{\unitlength}{0.0500bp}%
    \ifx\gptboxheight\undefined%
      \newlength{\gptboxheight}%
      \newlength{\gptboxwidth}%
      \newsavebox{\gptboxtext}%
    \fi%
    \setlength{\fboxrule}{0.5pt}%
    \setlength{\fboxsep}{1pt}%
\begin{picture}(7200.00,5040.00)%
    \gplgaddtomacro\gplbacktext{%
      \csname LTb\endcsname%%
      \put(1474,704){\makebox(0,0)[r]{\strut{}$0.0*10^{0}$}}%
      \put(1474,1292){\makebox(0,0)[r]{\strut{}$5.0*10^{-4}$}}%
      \put(1474,1880){\makebox(0,0)[r]{\strut{}$1.0*10^{-3}$}}%
      \put(1474,2468){\makebox(0,0)[r]{\strut{}$1.5*10^{-3}$}}%
      \put(1474,3055){\makebox(0,0)[r]{\strut{}$2.0*10^{-3}$}}%
      \put(1474,3643){\makebox(0,0)[r]{\strut{}$2.5*10^{-3}$}}%
      \put(1474,4231){\makebox(0,0)[r]{\strut{}$3.0*10^{-3}$}}%
      \put(1474,4819){\makebox(0,0)[r]{\strut{}$3.5*10^{-3}$}}%
      \put(1606,484){\makebox(0,0){\strut{}$0$}}%
      \put(2645,484){\makebox(0,0){\strut{}$1$}}%
      \put(3685,484){\makebox(0,0){\strut{}$2$}}%
      \put(4724,484){\makebox(0,0){\strut{}$3$}}%
      \put(5764,484){\makebox(0,0){\strut{}$4$}}%
      \put(6803,484){\makebox(0,0){\strut{}$5$}}%
    }%
    \gplgaddtomacro\gplfronttext{%
      \csname LTb\endcsname%%
      \put(308,2761){\rotatebox{-270}{\makebox(0,0){\strut{}Kehrwert der Zeit in $\si{\per \second}$}}}%
      \put(4204,154){\makebox(0,0){\strut{}Konzentration in  $\si{\mol \per \meter \tothe{3} }$}}%
      \csname LTb\endcsname%%
      \put(5870,4606){\makebox(0,0)[r]{\strut{}$1/T_{\text{1}}\left([\ce{Cu^2+}]\right)$}}%
      \csname LTb\endcsname%%
      \put(5870,4386){\makebox(0,0)[r]{\strut{}linearer Fit}}%
    }%
    \gplbacktext
    \put(0,0){\includegraphics{plots/Relaxivitat_CuT1}}%
    \gplfronttext
  \end{picture}%
\endgroup

    \caption{\textcolor{red}{ToD:}Relaxivitat $r_1$ von Kupfer}
    \label{fig:RelaxCUT1}
\end{figure}

\begin{figure}[H]
    \centering
    \input{plots/Relaxivitat_CuT2.tex}
    \caption{\textcolor{red}{ToD:}Relaxivitat $r_2$ von Kupfer}
    \label{fig:RelaxCUT2}
\end{figure}

\begin{figure}[H]
    \centering
    \input{plots/Relaxivitat_MnT1.tex}
    \caption{\textcolor{red}{ToD:}Relaxivitat $r_1$ von Mangan}
    \label{fig:RelaxMNT1}
\end{figure}

\begin{figure}[H]
    \centering
    % GNUPLOT: LaTeX picture with Postscript
\begingroup
  % Encoding inside the plot.  In the header of your document, this encoding
  % should to defined, e.g., by using
  % \usepackage[cp1252,<other encodings>]{inputenc}
  \inputencoding{cp1252}%
  \makeatletter
  \providecommand\color[2][]{%
    \GenericError{(gnuplot) \space\space\space\@spaces}{%
      Package color not loaded in conjunction with
      terminal option `colourtext'%
    }{See the gnuplot documentation for explanation.%
    }{Either use 'blacktext' in gnuplot or load the package
      color.sty in LaTeX.}%
    \renewcommand\color[2][]{}%
  }%
  \providecommand\includegraphics[2][]{%
    \GenericError{(gnuplot) \space\space\space\@spaces}{%
      Package graphicx or graphics not loaded%
    }{See the gnuplot documentation for explanation.%
    }{The gnuplot epslatex terminal needs graphicx.sty or graphics.sty.}%
    \renewcommand\includegraphics[2][]{}%
  }%
  \providecommand\rotatebox[2]{#2}%
  \@ifundefined{ifGPcolor}{%
    \newif\ifGPcolor
    \GPcolorfalse
  }{}%
  \@ifundefined{ifGPblacktext}{%
    \newif\ifGPblacktext
    \GPblacktexttrue
  }{}%
  % define a \g@addto@macro without @ in the name:
  \let\gplgaddtomacro\g@addto@macro
  % define empty templates for all commands taking text:
  \gdef\gplbacktext{}%
  \gdef\gplfronttext{}%
  \makeatother
  \ifGPblacktext
    % no textcolor at all
    \def\colorrgb#1{}%
    \def\colorgray#1{}%
  \else
    % gray or color?
    \ifGPcolor
      \def\colorrgb#1{\color[rgb]{#1}}%
      \def\colorgray#1{\color[gray]{#1}}%
      \expandafter\def\csname LTw\endcsname{\color{white}}%
      \expandafter\def\csname LTb\endcsname{\color{black}}%
      \expandafter\def\csname LTa\endcsname{\color{black}}%
      \expandafter\def\csname LT0\endcsname{\color[rgb]{1,0,0}}%
      \expandafter\def\csname LT1\endcsname{\color[rgb]{0,1,0}}%
      \expandafter\def\csname LT2\endcsname{\color[rgb]{0,0,1}}%
      \expandafter\def\csname LT3\endcsname{\color[rgb]{1,0,1}}%
      \expandafter\def\csname LT4\endcsname{\color[rgb]{0,1,1}}%
      \expandafter\def\csname LT5\endcsname{\color[rgb]{1,1,0}}%
      \expandafter\def\csname LT6\endcsname{\color[rgb]{0,0,0}}%
      \expandafter\def\csname LT7\endcsname{\color[rgb]{1,0.3,0}}%
      \expandafter\def\csname LT8\endcsname{\color[rgb]{0.5,0.5,0.5}}%
    \else
      % gray
      \def\colorrgb#1{\color{black}}%
      \def\colorgray#1{\color[gray]{#1}}%
      \expandafter\def\csname LTw\endcsname{\color{white}}%
      \expandafter\def\csname LTb\endcsname{\color{black}}%
      \expandafter\def\csname LTa\endcsname{\color{black}}%
      \expandafter\def\csname LT0\endcsname{\color{black}}%
      \expandafter\def\csname LT1\endcsname{\color{black}}%
      \expandafter\def\csname LT2\endcsname{\color{black}}%
      \expandafter\def\csname LT3\endcsname{\color{black}}%
      \expandafter\def\csname LT4\endcsname{\color{black}}%
      \expandafter\def\csname LT5\endcsname{\color{black}}%
      \expandafter\def\csname LT6\endcsname{\color{black}}%
      \expandafter\def\csname LT7\endcsname{\color{black}}%
      \expandafter\def\csname LT8\endcsname{\color{black}}%
    \fi
  \fi
    \setlength{\unitlength}{0.0500bp}%
    \ifx\gptboxheight\undefined%
      \newlength{\gptboxheight}%
      \newlength{\gptboxwidth}%
      \newsavebox{\gptboxtext}%
    \fi%
    \setlength{\fboxrule}{0.5pt}%
    \setlength{\fboxsep}{1pt}%
\begin{picture}(7200.00,5040.00)%
    \gplgaddtomacro\gplbacktext{%
      \csname LTb\endcsname%%
      \put(682,704){\makebox(0,0)[r]{\strut{}$0$}}%
      \put(682,1733){\makebox(0,0)[r]{\strut{}$5$}}%
      \put(682,2762){\makebox(0,0)[r]{\strut{}$10$}}%
      \put(682,3790){\makebox(0,0)[r]{\strut{}$15$}}%
      \put(682,4819){\makebox(0,0)[r]{\strut{}$20$}}%
      \put(814,484){\makebox(0,0){\strut{}$0$}}%
      \put(2012,484){\makebox(0,0){\strut{}$0.1$}}%
      \put(3210,484){\makebox(0,0){\strut{}$0.2$}}%
      \put(4407,484){\makebox(0,0){\strut{}$0.3$}}%
      \put(5605,484){\makebox(0,0){\strut{}$0.4$}}%
      \put(6803,484){\makebox(0,0){\strut{}$0.5$}}%
    }%
    \gplgaddtomacro\gplfronttext{%
      \csname LTb\endcsname%%
      \put(308,2761){\rotatebox{-270}{\makebox(0,0){\strut{}Kehrwert der Zeit in $\si{\frac{1}{\second}}$}}}%
      \put(3808,154){\makebox(0,0){\strut{}Konzentration in  $\si{\frac{\mol}{\meter \tothe{3}}}$}}%
      \csname LTb\endcsname%%
      \put(5858,4606){\makebox(0,0)[r]{\strut{}$1/T_{\text{2}}\left([\ce{Mn^2+}]\right)$}}%
      \csname LTb\endcsname%%
      \put(5858,4386){\makebox(0,0)[r]{\strut{}linearer Fit}}%
    }%
    \gplbacktext
    \put(0,0){\includegraphics{plots/Relaxivitat_MnT2}}%
    \gplfronttext
  \end{picture}%
\endgroup

    \caption{\textcolor{red}{ToD:}Relaxivitat $r_2$ von Mangan}
    \label{fig:RelaxMNT2}
\end{figure}

\begin{figure}[H]
    \centering
    \input{plots/KupferalleT1.tex}
    \caption{\textcolor{red}{ToD:}Alle Messungen T1 Cu2+}
    \label{fig:T1CU}
\end{figure}

\begin{figure}[H]
    \centering
    \input{plots/KupferalleT2.tex}
    \caption{\textcolor{red}{ToD:}Alle Messungen T2Cu2+}
    \label{fig:T2CU}
\end{figure}

\begin{figure}[H]
    \centering
    \input{plots/ManganalleT1.tex}
    \caption{\textcolor{red}{ToD:}Alle Messungen T1Mn2+}
    \label{fig:T1Mn}
\end{figure}

\begin{figure}[H]
    \centering
    \input{plots/ManganalleT2.tex}
    \caption{\textcolor{red}{ToD:}Alle Messungen T2MN2+}
    \label{fig:T2M}
\end{figure}

\begin{table}[H]
    	\centering
        \begin{tabular}{lllll}  \hline
        \multicolumn{1}{|l|}{}            & \multicolumn{1}{l|}{T1}      & \multicolumn{1}{l|}{U(T1)-Fit} & \multicolumn{1}{l|}{T2}      & \multicolumn{1}{l|}{U(T2)-Fit}  \\ \hline
        \multicolumn{1}{|l|}{Cu2    250}  & \multicolumn{1}{l|}{1394,84} & \multicolumn{1}{l|}{0,001055}  & \multicolumn{1}{l|}{1215,51} & \multicolumn{1}{l|}{0,0002529}  \\ \hline
        \multicolumn{1}{|l|}{Cu2    500}  & \multicolumn{1}{l|}{1003,4}  & \multicolumn{1}{l|}{0,0004851} & \multicolumn{1}{l|}{1066,44} & \multicolumn{1}{l|}{0,0002621}  \\ \hline
        \multicolumn{1}{|l|}{Cu2    1000} & \multicolumn{1}{l|}{646,849} & \multicolumn{1}{l|}{71,54}     & \multicolumn{1}{l|}{748,404} & \multicolumn{1}{l|}{0,0001937}  \\ \hline
        \multicolumn{1}{|l|}{Cu2    2000} & \multicolumn{1}{l|}{431,268} & \multicolumn{1}{l|}{0,0002906} & \multicolumn{1}{l|}{341,83}  & \multicolumn{1}{l|}{0,0001228}  \\ \hline
                                        &                              &                                &                              &                                 \\ \hline
        \multicolumn{1}{|l|}{}            & \multicolumn{1}{l|}{T1}      & \multicolumn{1}{l|}{U(T1)-Fit} & \multicolumn{1}{l|}{T2}      & \multicolumn{1}{l|}{U(T2)-Fit}  \\ \hline
        \multicolumn{1}{|l|}{Wasser}      & \multicolumn{1}{l|}{2199,46} & \multicolumn{1}{l|}{0,003027}  & \multicolumn{1}{l|}{1901,06} & \multicolumn{1}{l|}{89,83}      \\ \hline
                                        &                              &                                &                              &                                 \\
                                        &                              &                                &                              &                                 \\ \hline
        \multicolumn{1}{|l|}{}            & \multicolumn{1}{l|}{T1}      & \multicolumn{1}{l|}{U(T1)-Fit} & \multicolumn{1}{l|}{T2}      & \multicolumn{1}{l|}{U(T2)-Fit}  \\ \hline
        \multicolumn{1}{|l|}{Mn 2 25}     & \multicolumn{1}{l|}{1178,28} & \multicolumn{1}{l|}{0,0009801} & \multicolumn{1}{l|}{548,337} & \multicolumn{1}{l|}{0,0001258}  \\ \hline
        \multicolumn{1}{|l|}{Mn 2 50}     & \multicolumn{1}{l|}{725,857} & \multicolumn{1}{l|}{0,0006027} & \multicolumn{1}{l|}{279,858} & \multicolumn{1}{l|}{0,00008179} \\ \hline
        \multicolumn{1}{|l|}{Mn 2 100}    & \multicolumn{1}{l|}{316,085} & \multicolumn{1}{l|}{0,0003079} & \multicolumn{1}{l|}{170,996} & \multicolumn{1}{l|}{0,0001182}  \\ \hline
        \multicolumn{1}{|l|}{Mn 2 200}    & \multicolumn{1}{l|}{180,244} & \multicolumn{1}{l|}{0,0001274} & \multicolumn{1}{l|}{69,1512} & \multicolumn{1}{l|}{0,0000547}  \\ \hline
        \end{tabular}
        \caption{T1- und T2- abhängig von den Stoffen und der Konzentration}
\end{table}

\begin{figure}[H]
    \centering
    \input{plots/SignalkontrastT1.tex}
    \caption{Die T1 Signale bei der jeweiligen Konzentration}
    \label{fig:T1Signalkontrast}
\end{figure}

\begin{table}[H]
    \centering
    \caption{Relaxivitäten von Kupfer und Mangan.}
    \begin{tabular}{|l||r|r|} \hline
            & Kupfer    & Mangan    \\    \hline \hline
        $r_{1}$ in $\si{\frac{\mol}{\metre \cubed \second}}$    & $\SI{4.53 \pm 0.31e-4}{}$   & $\SI{1.365 \pm 0.084e-2}{}$    \\    \hline
        $T_{1}$ in $\si{\second}$                               & $\SI{1.84 \pm 0.24e3}{}$    & $\SI{5.7    \pm 6.3e3}{}$       \\    \hline
        $r_{2}$ in $\si{\frac{\mol}{\metre \cubed \second}}$    & $\SI{6.16 \pm 0.84e-4}{}$   & $\SI{3.593  \pm 0.0031e-1}{}$   \\    \hline
        $T_{2}$ in $\si{\second}$                               & $\SI{2.9  \pm 1.6e3}{}$     & $\SI{-3.2   \pm 7.5e3}{}$       \\    \hline
    \end{tabular} 
    \label{tab:Relaxivitat} 
\end{table}
    
    % \begin{figure}[H]
    %     \centering
    %     \input{plots/SignalkontrastT2.tex}
    %     \caption{Die T2 Signale bei der jeweiligen Konzentration}
    % \end{figure}
    %  ich glaube die Abbildung ist nicht gut genug, 
    % bzw. die Fehlerdiskussion hätte ich keine Ahnung, warum 1000mol in der Mitte von den zwei ist