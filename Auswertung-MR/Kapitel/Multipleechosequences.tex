% !TEX root = main.tex
\section{Multiple echo sequences}
\label{sec:Multipleechosequences}
Beside the \textit{Hahn} echo it is also possible to apply multiple \textit{Hahn} echos in one experimental measurement. This method is called \textit{Call-Purcell-Meiboom-Gill}-method (CPMG). Therefore the \SI{180}{\degree} pulse is applied every $2\tau$ and thus there accur many maximums in the signal every $2\tau$. The reason to use the CPMG method is that it is possible to measure the amplitude of two consecutiv maxima more often and therefore the measurement of T$_2$ is more precise. More about this will be discussed in the next chapter.\newline
To make the CPMG signals smoother in the time domain we do not use rectangular functions for the pulses, but smoothen them at the edge by a sine-bell-square function. This is possible, due to the fact that it does not change the physical properties of our measurements, but will make them smoother.\newline
A main advantage of CMPG is that errors in the refocusing pulse can be vanished (minimize term of inhomogenious magnetic field), by changing the phase between the B$_1$ excitation and the refocusing pulses. The program \textit{Prospa} profides a function called ''Constant 180 pulse phase''. This function keeps all the phases of the refocusing pulses equal. The second function \textit{Prospa} provides is ''Alternating 180 pulse phase''. This function compensates echo errors by alternating the refocusing pulses by \SI{180}{\degree}. In figure \ref{fig:180pulsephasedegree} it is visible what a change in the 180 pulse phase does to the signal. Unfortunately we only saved the signal vor 180 pulse phases of \SI{270}{\degree} and \SI{90}{\degree}. For those two values the signal does not change. That is also the reason why there is only one signal visible. The other one is just directly behind the other one and therefore not visible. If we would have saved a pulse phase of \SI{180}{\degree} the signal should change to \textcolor{red}{na zu was sollte es sich ändern?? schwächer?, was macht pulse phase eig genau}.

\begin{figure}[H]
    \centering
    \input{plots/180pulsephasedegree.tex}
    \caption[This figure shows the impact of the 180 pulse phase.]{This figure shows the impact of the 180 pulse phase. Unfortunately we only saved data for a 180 pulse phase of \SI{270}{\degree} and \SI{90}{\degree} and for those values it is correct that the signal does not change, but a signal for a 180 pulse phase of \SI{180}{\degree} would have shown a different signal.}
    \label{fig:180pulsephasedegree}
\end{figure}
