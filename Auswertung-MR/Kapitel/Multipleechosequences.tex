% !TEX root = main.tex
\section{Multiple echo sequences}
\label{sec:Multipleechosequences}

explain timedomain -> short discussion: fucntion (sine-bell-squared function? Section 5.5.3.2) to smoothen (because it doesnt change physics)
% \begin{figure}[H]
%     \centering
%     \input{plots/timedomain.tex}
%     \caption[]{time domain filter. it might be that at this picture both signals were taken with time domain filter on, because there should be a change in the shortness of the peaks, but it is not there.}
%     \label{fig:timedomain}
% \end{figure}
\begin{figure}[H]
    \centering
    \input{plots/180pulsephasedegree.tex}
    \caption[]{ask: what does pulse phase degree between \SI{90}{\degree} and \SI{180}{\degree} (or also between \SI{180}{\degree} and \SI{180}{\degree}) mean (Anleitung 9.)? -> minimize term of inhomogeniuos magnetic field; it is not the time between the pulses; phases\newline
    difference between alternating and constant \SI{180}{\degree} pulse phase-> alternating phase: computer does change phase degree; constant: manual change of phase degree-> look up manual for alternating; explain it\newline
    we only have data for 180 pulse phase degrees in \SI{270}{\degree} and \SI{90}{\degree}, but those two are the same and this is good, but we don´t hae values for \SI{180}{\degree} example. -> ask pther group for measuremtns at about 180 degree\newline
    we didn´t make measurements about 90 pulse phase degree}
    \label{fig:180pulsephasedegree}
\end{figure}
