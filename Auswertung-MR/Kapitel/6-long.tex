% !TEX root = main.tex
\section{Longitudinal relaxation measurements T1}
\label{sec:LongitudinalrelaxationmeasurementsT1}
There exist two possibilities to measure the longitudinal spin lattice relaxation.
First we want to have a closer look at the measurement via $\tau_p$ (polarizing pulse duration).
Therefore the computer program \textit{Prospa} applies a polarizing pulse orthogonal to the earths magnetic field.
Due to this polarizing pulse the spins align in the transversal plain and form a bulk magnetization.
By time the magnetization becomes stronger because of this the signal becomes stronger.
This relation is visualized in the figure \ref{fig:BildT1}.

\begin{figure}[H]
    \centering
    \includegraphics[width= \textwidth]{Abbildungen/BildT1.png}   
    \caption[Sketch to show how T$_1$ can be measured. \cite{Bild}]{Sketch to show how T$_1$ can be measured.
    One way is by changing the polarizing pulse duration $\tau_p$ and the other way is by varying the time between the polarizing pulse and the \SI{90}{\degree} pulse. \cite{Bild}}
    \label{fig:BildT1}
\end{figure}

Due to the increasing magnetization it is possible to calculate the T$_{1,p}$ relaxation.
In order to do so the magnetization time is increased step by step from \SI{500}{\milli \second} to \SI{4500}{\milli \second} with an increment of \SI{500}{\milli \second} and in each configuration the signals maximum is obtained from the fourier transformed spectrum.
Figure \ref{fig:T1Polarisationsfeldfeld} shows the attenuation of the signals normalized to the maximum peak E$_0$.
The underlying idea is that by applying a fit function as following:
\begin{align}
    S(x)=S_0 \cdot \left[1-\exp\left(\frac{-x}{T_{1,p}}\right)\right] \ ,
    \label{eq: fitBp}
\end{align}
it is possible to calculate the relaxation time T$_{1,p}$.
The exponential decay is a result of the loss of phase coherence between the spins and will be used for every measurement of spin relaxation.
In this case T$_{1,p}$ is found to count as \SI{2912.8800 \pm 0.0048}{\milli \second}.

\begin{figure}[H]
    \centering
    \input{plots/T1Polarisationsfeldfeld.tex}
    \caption[T$_{1,p}$ measurement by varying $\tau_p$ and observing how the attenuation $\frac{\text{E}}{\text{E}_0}$ evolves.]{T$_{1,p}$ measurement by varying $\tau_p$ and observing how the attenuation $\frac{\text{E}}{\text{E}_0}$ evolves.
    The provided exponential fit results in a value for T$_{1,p}$ of \SI{2912.8800 \pm 0.0048}{\milli \second}.}
    \label{fig:T1Polarisationsfeldfeld}
\end{figure}

The second option is to calculate the spin lattice relaxation via the earths magnetic field B$_e$.
In this case the index will be chosen as "$e$" for the spin lattice relaxation.
The procedure in this case is to change the time $t$ (pre-90 minimum delay) between the polarizing pulse ends and the \SI{90}{\degree} pulse begins.
This relation is also visualized in figure \ref{fig:BildT1}.
The pre-90 minimum delay is chosen as \SI{0}{\milli \second} and the pre-90 delay step size as \SI{500}{\milli \second}.
For every configuration the signal maximum is calculated again of the fourier transformed spectrum.
Figure \ref{fig:T1Erdmagnetfeld} shows the attenuation of the signals normalized to the maximum peak E$_0$.
This time the T$_{1,e}$ can be calculated by the following fit function:
\begin{align}
    S(x) = S_0 \cdot \exp\left(\frac{-x}{T_{1,e}}\right) \ .
    \label{eq: fitBe}
\end{align}
In our case T$_{1,e}$ is observed to be \SI{2753.0500 \pm 0.0012}{\milli \second}.\newline
In both ways the uncertainty of the T$_1$ values are quite small.
This is the result of really good aligned values to the fit function.
Nevertheless T$_{1,p}$ and T$_{1,e}$ are not consistent even though the uncertainty is considered.
This might be, due to the fact that those two measurements are based on two different methods and T$_{1,e}$ is dependent on the earths magnetic field.
Even though they are not consistent, the values for T$_{1,p}$ and T$_{1,e}$ have the same magnitude and also have the same magnitude according to a literature value of \SI{4000}{\milli \second} \cite{literaturT1}.
It is good to keep in mind that a comparison to literature values is just there to get the magnitude.
Since the surrounding magnetic field and the probe define the exact value.
\begin{figure}[H]
    \centering
    \input{plots/T1Erdmagnetfeld.tex}
    \caption[T$_{1,e}$ measurement by varying $t$ and observing how the attenuation $\frac{\text{E}}{\text{E}_0}$ evolves.]{T$_{1,e}$ measurement by varying $t$ and observing how the attenuation $\frac{\text{E}}{\text{E}_0}$ evolves.
    The provided exponential fit results in a value for T$_{1,e}$ of \SI{2753.0500 \pm 0.0012}{\milli \second}.}
    \label{fig:T1Erdmagnetfeld}
\end{figure}