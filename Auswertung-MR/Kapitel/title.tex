% !TEX root = main.tex
\title{Earths-Field-NMR Remote}
\subtitle{Physikalisches Fortgeschrittenenpraktikum at University of Constance}
\author{Authors: Philipp Gebauer, Simon Keegan and Marc Neumann \\ \large{Tutors: Narinder Narinder and Matthias Falk}}
\date{Execution on 9th and 27th of July 2020}
\maketitle
\begin{abstract}
    \begin{center}
        \Large{\textsf{\textbf{Abstract}}}
    \end{center}
    \vspace{0.75 cm}
    \begin{singlespace}
    \noindent The aim of this report is to show the principals of an EFNMR measurement and to discuss its results.\newline
    The first part of the experiment is about the basic principal of an EFNMR measurement.
    Therefore the noise level is taken into account and is identified to be \SI{7.5}{\mu \volt} for our setup.
    In order to tune the circuit to the lamor frequency of hydrogen at \SI{1841.4}{\hertz}, the LCR circuit in the B$_1$ coil has to have a capacity of \SI{13.8}{\nano \farad}.
    To obtain a sharp peak in the spectrum of the measured hydrogen signal the system was tuned to following values: shimming values $x = \SI{10.11}{\milli \ampere}, \ y = \SI{20.88}{\milli \ampere}, \ z = \SI{-20.07}{\milli \ampere}$; B$_1$ pulse duration \SI{1.35}{\milli \second}; capacity \SI{13.8}{\nano \farad}.
    The relaxation time measurements in the polarizing field results in values for T$_{1,p}$ of \SI{2912.8800 \pm 0.0048}{\milli \second}.
    The relaxation time measurements in the earths magnetic field results in values for T$_{1,e}$ of \SI{2753.0500 \pm 0.0012}{\milli \second}.
    The measurements of T$_2$ results in values of \SI{2691 \pm 12}{\milli \second} with single \textit{Hahn} echos and \SI{2317.76000 \pm 0.00062}{\milli \second} with the use of 30 echos in a CPMG.\\
    During the more application-oriented part of the experiment the relaxivities of copper and manganese were determined: $r_{1,\text{\ce{Cu^2+}}} = \SI{0.451 \pm 0.031}{\frac{\mol}{\metre \cubed \second}}$, $r_{2,\text{\ce{Cu^2+}}} = \SI{0.617 \pm 0.084}{\frac{\mol}{\metre \cubed \second}}$, $r_{1,\text{\ce{Mn^2+}}} = \SI{13.65 \pm 0.84}{\frac{\mol}{\metre \cubed \second}}$ and $r_{2,\text{\ce{Mn^2+}}} = \SI{35.9 \pm 3.1}{\frac{\mol}{\metre \cubed \second}}$. Those values reveal, that copper is well-suited as a contrast agent for experiments where $T_1$ is aimed to be stimulated whereas manganese is a good choice for $T_2$-orientated setups.
    Furthermore this part deals with the magnetic resonance imaging method and aims to get a deeper understanding how MRI works. Therefore pictures of a phantom were taken in 1D and 2D in order to characterize it and to analyse the content.

    Another target of the second part of the experiment is to determine a value for the diffusion coefficient by using the PGSE measurement method. With this method, the diffusion coefficient is determined to a value of $\SI{3,11(26)e-9}{\frac{\m^2}{s}}$
    \textcolor{red}{hier noch was zum zweiten teil schreiben!} 
    \vspace{0.75 cm}


    \noindent All authors have worked equally on all parts of this report and used no other sources than listed in the bibliography.

\end{singlespace}
\end{abstract}

\thispagestyle{empty}
\newpage