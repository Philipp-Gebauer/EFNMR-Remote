% !TEX root = main.tex
\title{Earths-Field-NMR Remote}
\subtitle{Physikalisches Fortgeschrittenenpraktikum at University of Constance}
\author{Authors: Philipp Gebauer, Simon Keegan and Marc Neumann \\ \large{Tutors: Narinder Narinder and Matthias Falk}}
\date{Execution on 9th of July 2020 and \textcolor{red}{???}}
\maketitle
\begin{abstract}
    \begin{center}
        \Large{\textsf{\textbf{Abstract}}}
    \end{center}
    \vspace{0.75 cm}
    \begin{singlespace}
    \noindent The aim of this paper is to show the principals of an EFNMR measurement and to discuss its results.\newline
    The first part of the experiment is about the basic principal of an EFNMR measurement. Therefore the noise level is taken into account and is identified to be \SI{7.5}{\mu \volt} for our setup. In order to tune the circuit to the lamor frequency of \SI{1841.4}{\hertz}, the LCR circuit in the B$_1$ coil has to have a capacity of \SI{13.8}{\nano \farad}. To obtain a sharp peak in the spectrum of the measured hydrogen signal the system was tuned to following values: shimming values $x = \SI{10.11}{\milli \ampere}, \ y = \SI{20.88}{\milli \ampere}, \ z = \SI{-20.07}{\milli \ampere}$; B$_1$ pulse duration \SI{1.35}{\milli \second}; capacity \SI{13.8}{\nano \farad}. The relaxation time measurements in the polarizing field results in values for T$_{1,p}$ of \SI{2912.8800 \pm 0.0048}{\milli \second}. The relaxation time measurements in the earths magnetic field results in values for T$_{1,e}$ of \SI{2753.0500 \pm 0.0012}{\milli \second}. The measurements of T$_2$ results in values of \SI{2691 \pm 12}{\milli \second} with single \textit{Hahn} echos and \SI{2317.76000 \pm 0.00062}{\milli \second} with the use of 30 echos in a CPMG.
    \vspace{0.75 cm}
     
    \noindent All authors have worked equally on all parts of this paper and used no other sources than listed in the bibliography.

\end{singlespace}
\end{abstract}

\thispagestyle{empty}
\newpage