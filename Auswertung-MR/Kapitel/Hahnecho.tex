% !TEX root = main.tex
\section{Hahn echo}
% \label{sec:Hahnecho}

\begin{figure}[H]
    \centering
    \input{plots/Echobeispeilsignal.tex}
    \caption[]{ask: wie safed no data for differnet $\tau$, is it ok just to explain it that the amplitude will decrease and the maximum will be shifted to a different time? -> yes
     this is an example for a hahn echo with shimming value \SI{4.95}{x}.}
    \label{fig:Echobeispeilsignal}
\end{figure}
\begin{figure}[H]
    \centering
    % GNUPLOT: LaTeX picture with Postscript
\begingroup
  % Encoding inside the plot.  In the header of your document, this encoding
  % should to defined, e.g., by using
  % \usepackage[cp1252,<other encodings>]{inputenc}
  \inputencoding{cp1252}%
  \makeatletter
  \providecommand\color[2][]{%
    \GenericError{(gnuplot) \space\space\space\@spaces}{%
      Package color not loaded in conjunction with
      terminal option `colourtext'%
    }{See the gnuplot documentation for explanation.%
    }{Either use 'blacktext' in gnuplot or load the package
      color.sty in LaTeX.}%
    \renewcommand\color[2][]{}%
  }%
  \providecommand\includegraphics[2][]{%
    \GenericError{(gnuplot) \space\space\space\@spaces}{%
      Package graphicx or graphics not loaded%
    }{See the gnuplot documentation for explanation.%
    }{The gnuplot epslatex terminal needs graphicx.sty or graphics.sty.}%
    \renewcommand\includegraphics[2][]{}%
  }%
  \providecommand\rotatebox[2]{#2}%
  \@ifundefined{ifGPcolor}{%
    \newif\ifGPcolor
    \GPcolorfalse
  }{}%
  \@ifundefined{ifGPblacktext}{%
    \newif\ifGPblacktext
    \GPblacktexttrue
  }{}%
  % define a \g@addto@macro without @ in the name:
  \let\gplgaddtomacro\g@addto@macro
  % define empty templates for all commands taking text:
  \gdef\gplbacktext{}%
  \gdef\gplfronttext{}%
  \makeatother
  \ifGPblacktext
    % no textcolor at all
    \def\colorrgb#1{}%
    \def\colorgray#1{}%
  \else
    % gray or color?
    \ifGPcolor
      \def\colorrgb#1{\color[rgb]{#1}}%
      \def\colorgray#1{\color[gray]{#1}}%
      \expandafter\def\csname LTw\endcsname{\color{white}}%
      \expandafter\def\csname LTb\endcsname{\color{black}}%
      \expandafter\def\csname LTa\endcsname{\color{black}}%
      \expandafter\def\csname LT0\endcsname{\color[rgb]{1,0,0}}%
      \expandafter\def\csname LT1\endcsname{\color[rgb]{0,1,0}}%
      \expandafter\def\csname LT2\endcsname{\color[rgb]{0,0,1}}%
      \expandafter\def\csname LT3\endcsname{\color[rgb]{1,0,1}}%
      \expandafter\def\csname LT4\endcsname{\color[rgb]{0,1,1}}%
      \expandafter\def\csname LT5\endcsname{\color[rgb]{1,1,0}}%
      \expandafter\def\csname LT6\endcsname{\color[rgb]{0,0,0}}%
      \expandafter\def\csname LT7\endcsname{\color[rgb]{1,0.3,0}}%
      \expandafter\def\csname LT8\endcsname{\color[rgb]{0.5,0.5,0.5}}%
    \else
      % gray
      \def\colorrgb#1{\color{black}}%
      \def\colorgray#1{\color[gray]{#1}}%
      \expandafter\def\csname LTw\endcsname{\color{white}}%
      \expandafter\def\csname LTb\endcsname{\color{black}}%
      \expandafter\def\csname LTa\endcsname{\color{black}}%
      \expandafter\def\csname LT0\endcsname{\color{black}}%
      \expandafter\def\csname LT1\endcsname{\color{black}}%
      \expandafter\def\csname LT2\endcsname{\color{black}}%
      \expandafter\def\csname LT3\endcsname{\color{black}}%
      \expandafter\def\csname LT4\endcsname{\color{black}}%
      \expandafter\def\csname LT5\endcsname{\color{black}}%
      \expandafter\def\csname LT6\endcsname{\color{black}}%
      \expandafter\def\csname LT7\endcsname{\color{black}}%
      \expandafter\def\csname LT8\endcsname{\color{black}}%
    \fi
  \fi
    \setlength{\unitlength}{0.0500bp}%
    \ifx\gptboxheight\undefined%
      \newlength{\gptboxheight}%
      \newlength{\gptboxwidth}%
      \newsavebox{\gptboxtext}%
    \fi%
    \setlength{\fboxrule}{0.5pt}%
    \setlength{\fboxsep}{1pt}%
\begin{picture}(7200.00,5040.00)%
    \gplgaddtomacro\gplbacktext{%
      \csname LTb\endcsname%%
      \put(682,704){\makebox(0,0)[r]{\strut{}$0$}}%
      \put(682,1253){\makebox(0,0)[r]{\strut{}$2$}}%
      \put(682,1801){\makebox(0,0)[r]{\strut{}$4$}}%
      \put(682,2350){\makebox(0,0)[r]{\strut{}$6$}}%
      \put(682,2899){\makebox(0,0)[r]{\strut{}$8$}}%
      \put(682,3447){\makebox(0,0)[r]{\strut{}$10$}}%
      \put(682,3996){\makebox(0,0)[r]{\strut{}$12$}}%
      \put(682,4545){\makebox(0,0)[r]{\strut{}$14$}}%
      \put(814,484){\makebox(0,0){\strut{}$1800$}}%
      \put(1563,484){\makebox(0,0){\strut{}$1810$}}%
      \put(2311,484){\makebox(0,0){\strut{}$1820$}}%
      \put(3060,484){\makebox(0,0){\strut{}$1830$}}%
      \put(3809,484){\makebox(0,0){\strut{}$1840$}}%
      \put(4557,484){\makebox(0,0){\strut{}$1850$}}%
      \put(5306,484){\makebox(0,0){\strut{}$1860$}}%
      \put(6054,484){\makebox(0,0){\strut{}$1870$}}%
      \put(6803,484){\makebox(0,0){\strut{}$1880$}}%
    }%
    \gplgaddtomacro\gplfronttext{%
      \csname LTb\endcsname%%
      \put(209,2761){\rotatebox{-270}{\makebox(0,0){\strut{}FID amplitude}}}%
      \put(3808,154){\makebox(0,0){\strut{}Frequency in $\si{\hertz}$}}%
      \csname LTb\endcsname%%
      \put(5816,4646){\makebox(0,0)[r]{\strut{}shimmin value \SI{0}{\milli \ampere} along x-axis}}%
      \csname LTb\endcsname%%
      \put(5816,4426){\makebox(0,0)[r]{\strut{}shimmin value \SI{4.95}{\milli \ampere} along x-axis}}%
    }%
    \gplbacktext
    \put(0,0){\includegraphics{plots/SpinEcho_4scans_ideal_Repetitiontime_0_shimming_150echo}}%
    \gplfronttext
  \end{picture}%
\endgroup

    \caption[]{ask: why are there different peaks a different shimming values?-> more frequency is sean (random) depends on position\newline
    which formula should we use to fit it? -> area under normalized spectrum should be the same; just discuss it -> narinder will send email
    
    for us: wieso signal schwächer-> mehr abweichung beim shimming (ursprünglich 10.11)-> abschwächung. integrale bei unterschiedlichen shimming; echo time 300ms bei beiden. }
    \label{fig:SpinEcho}
\end{figure}
We can measure T$_2$ when we don´t change the shimming values, because T$_2^*$ is dependent on a field inhomogeniousity. -> CPMG, Spin Hahn echo
