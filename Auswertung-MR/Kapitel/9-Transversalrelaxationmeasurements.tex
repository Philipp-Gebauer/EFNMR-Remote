% !TEX root = main.tex
\subsection{Transversal relaxation measurements}
\label{sec:Transversalrelaxationmeasurements}
The last chapter of the first part of this experiment concentrates on the transversal relaxation measurement.
In order to do so there are two possible ways again.\newline
The first one is by one single \textsc{Hahn} echo (spin echo).
Therefore the ratio between the maximum of the signal after the \SI{90}{\degree} pulse and the maximum after the echo (maximum after $2\tau$) provides the transversal relaxation time $T_2$.
Figure \ref{fig:T2} shows measurements for this method by different echo time steps of $2\cdot \SI{400}{\milli \second}$.
The exponential decay is clearly visible, due to the already explained loss of phase coherence between the spins.
Therefore to fit the datapoints the following formula has been used:
\begin{align}
    M(x)=M_0 \cdot \exp\left(\frac{-x}{T_{2}}\right) \ .
 \label{eq:T2}
\end{align}
This formula shows a $T_2$ relaxation time of \SI{2691 \pm 12}{\milli \second}.
It is important to remember that the phase coherence loss because of the spin-spin relaxation is irreversible and is always obtained when measuring $T_2$.

\begin{figure}[H]
    \centering
    \input{plots/T2.tex}
    \caption[Attenuation $\frac{\text{E}}{\text{E}_0}$ for different echo times and exponential fit of the data.]{Attenuation $\frac{\text{E}}{\text{E}_0}$ for different echo times and exponential fit of the data.
    The applied exponential fit results in a value of $T_2 = \SI{2691 \pm 12}{\milli \second}$.}
    \label{fig:T2}
\end{figure}

One disadvantage of the $T_2$ measurement via one single \textsc{Hahn} echo is that the ratio of two back to back maxima is not that decisive.
The second option to measure $T_2$ is by using CPMG. Now that more maxima can be observed, the ratio of back to back maxima can be calculated more precisely.
Therefore the result of $T_2$ is more accurate using this method.
Figure \ref{fig:CPMG} shows measured data for 30 different echos.
Due to the exponential decay formula \eqref{eq:T2} has been used again to fit the measured data.
This results in a value for $T_2$ of \SI{2317.76000 \pm 0.00062}{\milli \second}.
It is clearly visible that the uncertainty of this value is significantly lower than the value of the measurement via one single \textsc{Hahn} echo, therefore it is more exact.
A comparison to an example literature value of \SI{2000}{\milli \second} \cite{literaturT1} shows that the magnitude is correct.
Keeping in mind that a comparison to literature values is just there to get the magnitude the result is satisfying.
Since the surrounding magnetic field and the probe define the exact value as mentioned before.

\begin{figure}[H]
    \centering
    \input{plots/CPMG045shimming.tex}
    \caption[Attenuation $\frac{\text{E}}{\text{E}_0}$ for different echo maxima provided by the CPMG method.]{Attenuation $\frac{\text{E}}{\text{E}_0}$ for different echo maxima provided by the CPMG method.
    The applied exponential fit results in a value of $T_2 \SI{2317.76000 \pm 0.00062}{\milli \second}$.}
    \label{fig:CPMG}
\end{figure}

The difference of the two $T_2$ values might occur, due to de-shimming the system for the CPMG method and therefore some inaccurate pulse phases can have been measured.
Nevertheless note that the CPMG method is the more precise method to measure $T_2$, due to more back to back maxima.
By measuring $T_2$ via the \textsc{Hahn} echo and the CPMG the inhomogeneity of the magnetic field is reversed in theory, due to the \SI{180}{\degree} pulses.
Therefore they should not have a big impact on the relaxation time.
Nevertheless owing to inaccurate pulse phases, de-shimming can still have a small impact.
That is probably the reason why the two measured relaxation times do not take on the same value.