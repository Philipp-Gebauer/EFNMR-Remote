% !TEX root = main.tex
\subsection{J-Kopplung zur chemischen Strukturanalyse}
\begin{figure}[H]
    \centering
    \input{plots/JKopplung.tex}
    \caption{TODO!!;\\
    Zum tunen benutzte Daten des Puls and collect Experiments; Man soll noch die Integrale berechnen der einzelnen Peaks(aus den Integralen bekommt man dann das Verhältnuis von 1:2:3:3:2:1 oder so halt); Diese Peaks fitten bzw den abstand der Maximas berechnen=< daraus dann die Kopplungaskostante}
\end{figure}
