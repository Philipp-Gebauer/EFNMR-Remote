% !TEX root = main.tex
\section{Setup}
\label{sec:Aufbau}
This chapter is about the setup of this experiment. To understand which part of the experiment has what use, it is necessary to have a look at the components of the setup.\newline
Figure \ref{fig:Aufbau} shows the different coils which are necessary for the EFNMR measurement. The innerst coil B$_1$ is the excite and collect coil which is described in the previous chapter. The outer coil is to prepolarize the sample. This is necessary to obtain a stronger signal. By applying a strong magnetic field, all spins align in the direction of the prepolarising pulse and provides a bulk polarised nuclear magnetisation across the sample. The middle coil is the gradient coil. This coil erases the inhomogeneous magnetic field which always occurs for different uncertainty reasons. This coil is also used for the 2D imaging of the probe by adjusting the components of the magnetic field. The z-axis of the whole setup of the coils has to be aligned parallel to the earths magnetic field. Therefore a compass can be used to adjust the position. Via the computer program \textit{Prospa} and the spectrometer, the currents of the coil can be adjusted and the induced signals can be measured.
\begin{figure}[H]
    \centering
    \includegraphics[width= \textwidth]{Abbildungen/Aufbau.png}   
    \caption[Setup of the \textit{Terranova-MRI EFNMR}. \cite{Bild}]{Setup of the \textit{Terranova-MRI EFNMR}. On the left handside the coils B$_1$ (excite and collect coil), gradient coil (homogeneous magnetic field and 2D scanning coil) and the prepolarising coil B$_p$ are seen. The right hand side shows the water sample which has been used and the spectrometer which adjusts the necessary signals to the coils. \cite{Bild}}
    \label{fig:Aufbau}
\end{figure}
