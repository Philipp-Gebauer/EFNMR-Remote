% !TEX root = main.tex
\section{Error discussion and conclusion}
\label{sec:Fazit}
Despite all measured data make sense and are consistent with the literature, there are also errors which occured. Lets begin by the setup and its position itself. Due to the orientation and position in the room the noise level in chapter \ref{sec:Noisemeasurement} changes a lot. The noise level \SI{7.5}{\mu \volt} is a value below \SI{10}{\mu \volt} and therefore a acceptable noise level. To achieve excellent results it is even though necessary to have a noise level below \SI{3}{\mu \volt}. Furthermore some of the measured data depend on the earths magnetic field and since this is not really strong (magnitude \si{\nano \tesla}) even small metal objects can change its properties. The influence of the light source and the computer display were already discussed in the chapter \ref{sec:Noisemeasurement}. Even though the gradient coil should erase all inhomogeneous components of the magnetic field, there is always a slight propability that it is not homogeneous.\newline
Another imperfection of the experiment is the duration of the pulses and the phases. With the help of figure \ref{fig:B1dauer} we analyzed that the B$_1$ duration for a \SI{90}{\degree} pulse is \SI{1.35}{\milli \second}. Since the measurement steps in the figure \ref{fig:B1dauer} are rather high, the duration could also change a little. This error has a huge impact when it comes to the \textit{Hahn} echo and the CPMG method and thus changes T$_2$.
\\
Despite all errors the results confirm the theorey and are consistent with the literature. This is allocable by the values for T$_{1,p}$ of \SI{2912.8800 \pm 0.0048}{\milli \second} and for T$_{1,e}$ of \SI{2753.0500 \pm 0.0012}{\milli \second} and for the measurements of T$_2$ of \SI{2317.76000 \pm 0.00062}{\milli \second}. The research of the hydrogen signal in a pulse and collect experiment results in a width of the peak at half maximum (FWHM) of \SI{1.177 \pm 0.042}{\hertz} and is therefore really small. With the help of those values, the discussion of the values and the detailed introduction, the experiment shows the properties of a basic ENMR experiment quite good.\\
During the application-orientated part of the experiment the influence of contrast agents on the relaxation times $T_1$ and $T_2$ have been observed. Despite not having saved all acquired data several results were determined and support the in advance made assumptions.
First of all the influence of different polarizing times on the signal intensity were discussed as no data was saved. Then the relaxation times $T_1 \approx \SI{2.2}{\second}$ and $T_2 \approx \SI{1.9}{\second}$ have been obtained from curves following the concepts described in part \ref{sec:PartI}. Afterwards the same procedure was done for eight different water samples mixed with different amounts of the substances copper and manganese. As a result of that the relaxivities $r_{\text{1,c}} = \SI{0.454 \pm 0.031}{\mole \per \metre \tothe{3} \per \second}$, $r_{\text{2,c}} = \SI{0.617 \pm 0.084}{\mole \per \metre \tothe{3} \per \second}$, $r_{\text{1,m}} = \SI{13.65 \pm 0.84}{\mole \per \metre \tothe{3} \per \second}$ and $r_{\text{1,c}} = \SI{35.9 \pm 3.1}{\mole \per \metre \tothe{3} \per \second}$ were calculated. 
Those indicate that copper is well-suited as a contrast agent for experiments where $T_1$ is aimed to be stimulated whereas manganese is a good choice for $T_2$-orientated setups.
It has to be mentioned that the occuring uncertainties have quite large values, especially in the case of manganese.
Additionally the obtianed values of $T_1(0)$ and $T_2(0)$ do not perfectly match the relaxation times calculated at the beginning.
That leads to conclusion, that the calculated values should not be taken for granted, but the occuring trends are obvious to recognize.\\
The aim of the 1D-MRI and 2D-MRI is to get a better understanding of how the imaging method exactly works and to determine the size and the form of the object. In the 1D-Imaging this lead to a guess of an cylindrical shape of the phantom with a volume of $\SI{532}{\milli\liter}$, which is near the $\SI{500}{\milli\liter}$ of the other probes. To get a better result in this experiment, there are some options to lower the uncertainties. One method is to change parameters during the measurement process like the FOV or the numbers of scans. With this the resolution of the measurement can be increased or decreased. It is also important to watch on the duration of the measurement to get a good efficiency out of it. \\
That leads to conclusion, that the calculated values should not be taked for granted, but the occuring trends are obvious to recognize.
The aim of the 1D-MRI and 2D-MRI is to get a better understanding of how the imaging method exactly works and to determine the size and the form of the object. In the 1D-Imaging this lead to a guess of an cylindrical shape of the phantom with a volume of $\SI{532}{\milli\liter}$, which is near the $\SI{500}{\milli\liter}$ of the other probes. To get a better result in this experiment, there are some options to lower the uncertainties. One method is to change parameters during the measurement process like the FOV or the numbers of scans. With this the resolution of the measurement can be increased or decreased. Also it is important to watch on the duration of the measurement to get a good efficiency out of it. \\
This is especially in the 2D-MRI important, because the measurement is not only in one dimension. In this part, many pictures were taken and compared to understand how positive or negative contrasting works with tuning the parameters of the polarisation time and the echo time. With the help of the pictures, the liquids in each of the compartment of the tubes can be guessed. In the left tube, it is expected to be water with a $T_1=\SI{1901,06}{\milli\s}$ and in the other compartment it is \ce{Cu2+} with a concentration with $\SI{500}{\micro\mole}$ or 
$\SI{1000}{\micro\mole}$ or the other possibility is \ce{Mn2+} in the compartment with a concentration of $\SI{50}{\micro\mole}$. To determine this exactly, it is important to measure the $T_1$ and $T_2$ independent of the 2D-MRI.  \\
During the structural analysis of difluorobenzene the coupling constant of the J-coupling of the atoms was calculated and shows a vlaue of $J = \SI{6.3 \pm 1.3}{\hertz}$.
That underlines good accordance to the theoretically predicted value of $J= \SI{6}{\hertz}$, even if the calculated uncertainty has a quite large value which nevertheless lies in the same order of magnitude.
Besides that the hypothesis of weak coupling was disapproved as $J$ does not have a significant lower value than the difference of the position of the main maxima of the fluor and hydrogen peaks in the frequency spectrum.
That supports the finding made during the analysis of the peak ratios, that those peaks seem to be broadened. 
This fact can occur when there is no weak coupling and second order terms would need to be considered.\\
For the last part of the experiment, it is possible to compare the self diffusion coefficient of $\SI{3,11(26)e-9}{\frac{\m^2}{s}}$ with the literature value of $2,299\times  10^{-9}\si{\frac{\m^2}{\s}}$. The magnitude of these two values are the same. The difference between these two factors are the pre-factor, which have got a deviation of $\pm1$. Considering, which uncertainties could occur during the experiment, this is a good value for the self diffusion coefficient. There are some methods to get a better measurement in this experiment, if it is necessary. \\  
One method to do this, is to use a sponge in the probe so that there are no fluctuation in the fluid any more. As a result of that, only the self diffusion will be measured.\\
Another important thing is, that the temperature has to be same during the experiment and especially has to be noted, so it is possible to compare the measured value with the literature values. 