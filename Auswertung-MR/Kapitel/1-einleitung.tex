% !TEX root = main.tex
\section{Introduction}
\label{sec:Introduction}
Earths field nuclear magnetic resonance is a widely used method in the quality management or in medical technology to gain knowledge about the structure of materials.
Therefore the magnetic moment of spins is taken into account.\newline
Due to the external magnetic field of the earth B$_0$ (sometimes refered to as B$_e$) the spins of hydrogen (spin quantum number: $I=\frac{1}{2}$) align either parallel or antiparallel to this magnetic field.
Using the \textit{Boltzmann} statistics it can be calculated, if the spins are aligned parallel or antiparallel.
Therefore the information about the temperature and the surrounding magnetic field B$_0$ is necessary.
Each spin precesses around the surrounding magnetic field B$_0$ (along z-axis), most of the time in the spin up direction, because it is energetically more favorable.
This precession evokes a component of the spins in the transversal plane.
However, since the phase of the precession is random the net magentisation is aligned along the z-axis.
By changing the properties of the surrounding magnetic field, the bulk magnetisation vector can be manipulated.
In order to do so an alternating electro magnetic field pulse is applied.
The frequency of this magnetic field (B$_1$) is in the radio frequency (RF) magnitude for large B$_0$ and for low B$_0$ it is in the ultra low frequency (ULF).
Since we use the earths magnetic field B$_e$ for our measurements, the frequency is in the ULF magnitude.
When the frequency of this magnetic field pulse is chosen right at the larmor frequency of the sample, the transitions between the energy levels of spin up and down happen more likely and therefore a phase coherence of the spins occurs.
The applied pulse results in changing the spins direction by a tipping angle $\Theta$ from the vertical to the transversal plane.
The precession of the spins can be measured in the transversal plane by a coil (B$_1$ coil) which is aligned orthogonal to the earths magnetic field.
The B$_1$ coil is therefore the exciting and detecting coil and therefore the heart of our measurements. \\
The first part of this experiment is about the basics of ENMR.
At first we have a look at the noise that is dependent on surrounding metal objects.
Then we analyse the B$_1$ coil by changing the capacity of the LCR circuit.
The next step is the optimization and characterisation of a free induction decay (FID) of a water sample.
The aim of this chapter is to measure a sharp peak at the larmor frequency of the hydrogen in the water.
When this is done the longitudinal relaxation time T$_1$ and the transversal relaxation time T$_2$ are measured and discussed.
\textcolor{red}{hier noch was zum zweiten teil schreiben!}