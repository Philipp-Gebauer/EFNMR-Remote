% !TEX root = main.tex
\subsection{\textcolor{red}{ToDo!}Relaxationskontrast}
\label{sec:Signalintensitaet}
Da der zweite Teil des Experiments an einem anderen Versuchstag durchgeführt wurde, wurde zu Beginn erneut eine Optimierung des Aufbaus auf die Wasserprobe vorgenommen um die Vergleichbarkeit der Ergebnisse der beiden Versuchstage zu schaffen. Dies geschah analog zum Vorgehen, welches in Abschnitt \ref{sec:OptimizationandCharacterisationofFIDinwatersample} dargelegt wurde.

Der Versuchsteil selbst zielt darauf ab zu untersuchen, wie longitudinale ($T_1$) und transversale ($T_2$) Relaxationszeit durch paramagnetische Ionen beeinflusst werden können. 
Weiterführend kann dadurch -- abhängig von $T_1$ und $T_2$ -- der Kontrast innerhalb eines bildgebenden Verfahrens verändert werden und somit können verschiedene Bereiche einer Probe besser aufgelöst und analysiert werden. 
Daher sollen die sogenannte Relaxivitäten $r_1$ und $r_2$ nach
\begin{align}
    \frac{1}{T_{\text{i}}\left([\ce{X^2+}]\right)} = r_{\text{i}} \cdot [\ce{X^2+}] + \frac{1}{T_{\text{i}}(0)} \label{eq:relaxivitat}
\end{align}
bestimmt werden. Dabei bezeichnet $T_{\text{i}}\left([\ce{X^2+}]\right)$ die -- von der Konzentration des verwendeten paramagnetischen Salzes im Wasser abhängige -- jeweilige Relaxationszeit, $[\ce{X^2+}]$  die entsprechende Ionenkonzentration und $T_{\text{i}}(0)$ die Relaxtionszeit im Falle, dass kein Kontrastmittel verwendet wird.




\begin{figure}[H]
    \centering
    % GNUPLOT: LaTeX picture with Postscript
\begingroup
  % Encoding inside the plot.  In the header of your document, this encoding
  % should to defined, e.g., by using
  % \usepackage[cp1252,<other encodings>]{inputenc}
  \inputencoding{cp1252}%
  \makeatletter
  \providecommand\color[2][]{%
    \GenericError{(gnuplot) \space\space\space\@spaces}{%
      Package color not loaded in conjunction with
      terminal option `colourtext'%
    }{See the gnuplot documentation for explanation.%
    }{Either use 'blacktext' in gnuplot or load the package
      color.sty in LaTeX.}%
    \renewcommand\color[2][]{}%
  }%
  \providecommand\includegraphics[2][]{%
    \GenericError{(gnuplot) \space\space\space\@spaces}{%
      Package graphicx or graphics not loaded%
    }{See the gnuplot documentation for explanation.%
    }{The gnuplot epslatex terminal needs graphicx.sty or graphics.sty.}%
    \renewcommand\includegraphics[2][]{}%
  }%
  \providecommand\rotatebox[2]{#2}%
  \@ifundefined{ifGPcolor}{%
    \newif\ifGPcolor
    \GPcolorfalse
  }{}%
  \@ifundefined{ifGPblacktext}{%
    \newif\ifGPblacktext
    \GPblacktexttrue
  }{}%
  % define a \g@addto@macro without @ in the name:
  \let\gplgaddtomacro\g@addto@macro
  % define empty templates for all commands taking text:
  \gdef\gplbacktext{}%
  \gdef\gplfronttext{}%
  \makeatother
  \ifGPblacktext
    % no textcolor at all
    \def\colorrgb#1{}%
    \def\colorgray#1{}%
  \else
    % gray or color?
    \ifGPcolor
      \def\colorrgb#1{\color[rgb]{#1}}%
      \def\colorgray#1{\color[gray]{#1}}%
      \expandafter\def\csname LTw\endcsname{\color{white}}%
      \expandafter\def\csname LTb\endcsname{\color{black}}%
      \expandafter\def\csname LTa\endcsname{\color{black}}%
      \expandafter\def\csname LT0\endcsname{\color[rgb]{1,0,0}}%
      \expandafter\def\csname LT1\endcsname{\color[rgb]{0,1,0}}%
      \expandafter\def\csname LT2\endcsname{\color[rgb]{0,0,1}}%
      \expandafter\def\csname LT3\endcsname{\color[rgb]{1,0,1}}%
      \expandafter\def\csname LT4\endcsname{\color[rgb]{0,1,1}}%
      \expandafter\def\csname LT5\endcsname{\color[rgb]{1,1,0}}%
      \expandafter\def\csname LT6\endcsname{\color[rgb]{0,0,0}}%
      \expandafter\def\csname LT7\endcsname{\color[rgb]{1,0.3,0}}%
      \expandafter\def\csname LT8\endcsname{\color[rgb]{0.5,0.5,0.5}}%
    \else
      % gray
      \def\colorrgb#1{\color{black}}%
      \def\colorgray#1{\color[gray]{#1}}%
      \expandafter\def\csname LTw\endcsname{\color{white}}%
      \expandafter\def\csname LTb\endcsname{\color{black}}%
      \expandafter\def\csname LTa\endcsname{\color{black}}%
      \expandafter\def\csname LT0\endcsname{\color{black}}%
      \expandafter\def\csname LT1\endcsname{\color{black}}%
      \expandafter\def\csname LT2\endcsname{\color{black}}%
      \expandafter\def\csname LT3\endcsname{\color{black}}%
      \expandafter\def\csname LT4\endcsname{\color{black}}%
      \expandafter\def\csname LT5\endcsname{\color{black}}%
      \expandafter\def\csname LT6\endcsname{\color{black}}%
      \expandafter\def\csname LT7\endcsname{\color{black}}%
      \expandafter\def\csname LT8\endcsname{\color{black}}%
    \fi
  \fi
    \setlength{\unitlength}{0.0500bp}%
    \ifx\gptboxheight\undefined%
      \newlength{\gptboxheight}%
      \newlength{\gptboxwidth}%
      \newsavebox{\gptboxtext}%
    \fi%
    \setlength{\fboxrule}{0.5pt}%
    \setlength{\fboxsep}{1pt}%
\begin{picture}(7200.00,5040.00)%
    \gplgaddtomacro\gplbacktext{%
      \csname LTb\endcsname%%
      \put(682,704){\makebox(0,0)[r]{\strut{}$0$}}%
      \put(682,1253){\makebox(0,0)[r]{\strut{}$10$}}%
      \put(682,1801){\makebox(0,0)[r]{\strut{}$20$}}%
      \put(682,2350){\makebox(0,0)[r]{\strut{}$30$}}%
      \put(682,2899){\makebox(0,0)[r]{\strut{}$40$}}%
      \put(682,3447){\makebox(0,0)[r]{\strut{}$50$}}%
      \put(682,3996){\makebox(0,0)[r]{\strut{}$60$}}%
      \put(682,4545){\makebox(0,0)[r]{\strut{}$70$}}%
      \put(814,484){\makebox(0,0){\strut{}$1820$}}%
      \put(1563,484){\makebox(0,0){\strut{}$1825$}}%
      \put(2311,484){\makebox(0,0){\strut{}$1830$}}%
      \put(3060,484){\makebox(0,0){\strut{}$1835$}}%
      \put(3809,484){\makebox(0,0){\strut{}$1840$}}%
      \put(4557,484){\makebox(0,0){\strut{}$1845$}}%
      \put(5306,484){\makebox(0,0){\strut{}$1850$}}%
      \put(6054,484){\makebox(0,0){\strut{}$1855$}}%
      \put(6803,484){\makebox(0,0){\strut{}$1860$}}%
    }%
    \gplgaddtomacro\gplfronttext{%
      \csname LTb\endcsname%%
      \put(308,2761){\rotatebox{-270}{\makebox(0,0){\strut{}Amplitude in willk\"urlicher Einheit}}}%
      \put(3808,154){\makebox(0,0){\strut{}Frequenz in $\si{\hertz}$}}%
      \csname LTb\endcsname%%
      \put(5858,4606){\makebox(0,0)[r]{\strut{}Polarisationszeit $\SI{4.0}{\second}$}}%
      \csname LTb\endcsname%%
      \put(5858,4386){\makebox(0,0)[r]{\strut{}Polarisationszeit $\SI{0.5}{\second}$}}%
    }%
    \gplbacktext
    \put(0,0){\includegraphics{plots/Polarisationszeit}}%
    \gplfronttext
  \end{picture}%
\endgroup

    \caption[Abhängigkeit der Signalintensität von der Polarisationszeit.]{In dieser Abbildung sind zwei ,,Pulse and Collect''-Messungen der bidestillierten Wasserprobe dargestellt. Die rote Kurve zeigt eine Messung mit einer Polarisationszeit von $\SI{4}{\second}$, bei der blauen Kurve beträgt die Polarisationszeit $\SI{0.5}{\second}$. Hierbei ist deutlich zu erkennen, dass die Signalintensität durch die höhere Polarisationszeit deutlich gesteigert werden kann. Bei den dargestellten Messungen um circa Faktor drei.} 
    \label{fig:SignalintensitaetPolarisationszeit}    
\end{figure}







%--------------------------------------------------------------
%-----------------noch zu verarbeiten--------------------------
%--------------------------------------------------------------

\ce{Cu^2+} \ce{Mn^2+}

\begin{figure}[H]
    \centering
    \input{plots/T1Wasser.tex}
    \caption{T1 Messung von Wasser}
\end{figure}

\begin{figure}[H]
    \centering
    \input{plots/T2Wasser.tex}
    \caption{T2 Messung von Wasser}
\end{figure}

\begin{figure}[H]
    \centering
    % GNUPLOT: LaTeX picture with Postscript
\begingroup
  % Encoding inside the plot.  In the header of your document, this encoding
  % should to defined, e.g., by using
  % \usepackage[cp1252,<other encodings>]{inputenc}
  \inputencoding{cp1252}%
  \makeatletter
  \providecommand\color[2][]{%
    \GenericError{(gnuplot) \space\space\space\@spaces}{%
      Package color not loaded in conjunction with
      terminal option `colourtext'%
    }{See the gnuplot documentation for explanation.%
    }{Either use 'blacktext' in gnuplot or load the package
      color.sty in LaTeX.}%
    \renewcommand\color[2][]{}%
  }%
  \providecommand\includegraphics[2][]{%
    \GenericError{(gnuplot) \space\space\space\@spaces}{%
      Package graphicx or graphics not loaded%
    }{See the gnuplot documentation for explanation.%
    }{The gnuplot epslatex terminal needs graphicx.sty or graphics.sty.}%
    \renewcommand\includegraphics[2][]{}%
  }%
  \providecommand\rotatebox[2]{#2}%
  \@ifundefined{ifGPcolor}{%
    \newif\ifGPcolor
    \GPcolorfalse
  }{}%
  \@ifundefined{ifGPblacktext}{%
    \newif\ifGPblacktext
    \GPblacktexttrue
  }{}%
  % define a \g@addto@macro without @ in the name:
  \let\gplgaddtomacro\g@addto@macro
  % define empty templates for all commands taking text:
  \gdef\gplbacktext{}%
  \gdef\gplfronttext{}%
  \makeatother
  \ifGPblacktext
    % no textcolor at all
    \def\colorrgb#1{}%
    \def\colorgray#1{}%
  \else
    % gray or color?
    \ifGPcolor
      \def\colorrgb#1{\color[rgb]{#1}}%
      \def\colorgray#1{\color[gray]{#1}}%
      \expandafter\def\csname LTw\endcsname{\color{white}}%
      \expandafter\def\csname LTb\endcsname{\color{black}}%
      \expandafter\def\csname LTa\endcsname{\color{black}}%
      \expandafter\def\csname LT0\endcsname{\color[rgb]{1,0,0}}%
      \expandafter\def\csname LT1\endcsname{\color[rgb]{0,1,0}}%
      \expandafter\def\csname LT2\endcsname{\color[rgb]{0,0,1}}%
      \expandafter\def\csname LT3\endcsname{\color[rgb]{1,0,1}}%
      \expandafter\def\csname LT4\endcsname{\color[rgb]{0,1,1}}%
      \expandafter\def\csname LT5\endcsname{\color[rgb]{1,1,0}}%
      \expandafter\def\csname LT6\endcsname{\color[rgb]{0,0,0}}%
      \expandafter\def\csname LT7\endcsname{\color[rgb]{1,0.3,0}}%
      \expandafter\def\csname LT8\endcsname{\color[rgb]{0.5,0.5,0.5}}%
    \else
      % gray
      \def\colorrgb#1{\color{black}}%
      \def\colorgray#1{\color[gray]{#1}}%
      \expandafter\def\csname LTw\endcsname{\color{white}}%
      \expandafter\def\csname LTb\endcsname{\color{black}}%
      \expandafter\def\csname LTa\endcsname{\color{black}}%
      \expandafter\def\csname LT0\endcsname{\color{black}}%
      \expandafter\def\csname LT1\endcsname{\color{black}}%
      \expandafter\def\csname LT2\endcsname{\color{black}}%
      \expandafter\def\csname LT3\endcsname{\color{black}}%
      \expandafter\def\csname LT4\endcsname{\color{black}}%
      \expandafter\def\csname LT5\endcsname{\color{black}}%
      \expandafter\def\csname LT6\endcsname{\color{black}}%
      \expandafter\def\csname LT7\endcsname{\color{black}}%
      \expandafter\def\csname LT8\endcsname{\color{black}}%
    \fi
  \fi
    \setlength{\unitlength}{0.0500bp}%
    \ifx\gptboxheight\undefined%
      \newlength{\gptboxheight}%
      \newlength{\gptboxwidth}%
      \newsavebox{\gptboxtext}%
    \fi%
    \setlength{\fboxrule}{0.5pt}%
    \setlength{\fboxsep}{1pt}%
\begin{picture}(7200.00,5040.00)%
    \gplgaddtomacro\gplbacktext{%
      \csname LTb\endcsname%%
      \put(1474,704){\makebox(0,0)[r]{\strut{}$0.0*10^{0}$}}%
      \put(1474,1292){\makebox(0,0)[r]{\strut{}$5.0*10^{-4}$}}%
      \put(1474,1880){\makebox(0,0)[r]{\strut{}$1.0*10^{-3}$}}%
      \put(1474,2468){\makebox(0,0)[r]{\strut{}$1.5*10^{-3}$}}%
      \put(1474,3055){\makebox(0,0)[r]{\strut{}$2.0*10^{-3}$}}%
      \put(1474,3643){\makebox(0,0)[r]{\strut{}$2.5*10^{-3}$}}%
      \put(1474,4231){\makebox(0,0)[r]{\strut{}$3.0*10^{-3}$}}%
      \put(1474,4819){\makebox(0,0)[r]{\strut{}$3.5*10^{-3}$}}%
      \put(1606,484){\makebox(0,0){\strut{}$0$}}%
      \put(2645,484){\makebox(0,0){\strut{}$1$}}%
      \put(3685,484){\makebox(0,0){\strut{}$2$}}%
      \put(4724,484){\makebox(0,0){\strut{}$3$}}%
      \put(5764,484){\makebox(0,0){\strut{}$4$}}%
      \put(6803,484){\makebox(0,0){\strut{}$5$}}%
    }%
    \gplgaddtomacro\gplfronttext{%
      \csname LTb\endcsname%%
      \put(308,2761){\rotatebox{-270}{\makebox(0,0){\strut{}Kehrwert der Zeit in $\si{\per \second}$}}}%
      \put(4204,154){\makebox(0,0){\strut{}Konzentration in  $\si{\mol \per \meter \tothe{3} }$}}%
      \csname LTb\endcsname%%
      \put(5870,4606){\makebox(0,0)[r]{\strut{}$1/T_{\text{1}}\left([\ce{Cu^2+}]\right)$}}%
      \csname LTb\endcsname%%
      \put(5870,4386){\makebox(0,0)[r]{\strut{}linearer Fit}}%
    }%
    \gplbacktext
    \put(0,0){\includegraphics{plots/Relaxivitat_CuT1}}%
    \gplfronttext
  \end{picture}%
\endgroup

    \caption[]{\textcolor{red}{ToD:}Diese Abbildung zeigt die Kehrwerte der vier Relaxationszeiten $T_1$ aufgetragen über den entsprechenden Konzentrationen des im Wasser gelösten Kupersalzes. Mittels eines linearen Fits konnten somit die Relaxivität $r_1$, sowie die Relaxationszeit ohne gelöstes Kontrastmittel nach Gleichung \eqref{eq:relaxivitat} ermittlet werden.
    
    Relaxivitat $r_1$ von Kupfer}
    \label{fig:RelaxCUT1}
\end{figure}

\begin{figure}[H]
    \centering
    \input{plots/KupferalleT1.tex}
    \caption{\textcolor{red}{ToD:}Alle Messungen T1 Cu2+}
    \label{fig:T1CU}
\end{figure}

\begin{figure}[H]
    \centering
    \input{plots/KupferalleT2.tex}
    \caption{\textcolor{red}{ToD:}Alle Messungen T2Cu2+}
    \label{fig:T2CU}
\end{figure}

\begin{figure}[H]
    \centering
    \input{plots/ManganalleT1.tex}
    \caption{\textcolor{red}{ToD:}Alle Messungen T1Mn2+}
    \label{fig:T1Mn}
\end{figure}

\begin{figure}[H]
    \centering
    \input{plots/ManganalleT2.tex}
    \caption{\textcolor{red}{ToD:}Alle Messungen T2MN2+}
    \label{fig:T2M}
\end{figure}

\begin{figure}[H]
    \centering
    \input{plots/SignalkontrastT1.tex}
    \caption{Die T1 Signale bei der jeweiligen Konzentration}
    \label{fig:T1Signalkontrast}
\end{figure}

\begin{table}[H]
    \centering
    \caption{Relaxivitäten von Kupfer und Mangan.}
    \begin{tabular}{|l||r|r|r|r|} \hline
        Kontrastmittel & $r_{1}$ in $\si{\frac{\mol}{\metre \cubed \second}}$    &  $T_{1}$ in $\si{\second}$ & $r_{2}$ in $\si{\frac{\mol}{\metre \cubed \second}}$ & $T_{2}$ in $\si{\second}$  \\ \hline \hline
        Kupfer & $\SI{0.454 \pm 0.031}{}$   & $\SI{1.84 \pm 0.24}{}$    & $\SI{0.617 \pm 0.084}{}$   & $\SI{2.9  \pm 1.6}{}$ \\ \hline
        Mangan & $\SI{13.65 \pm 0.84}{}$    & $\SI{5.7 \pm 6.3}{}$      & $\SI{35.9 \pm 3.1}{}$  & $\SI{-3.2 \pm 7.5}{}$ \\ \hline
    \end{tabular} 
    \label{tab:Relaxivitat} 
\end{table}
    
    % \begin{figure}[H]
    %     \centering
    %     \input{plots/SignalkontrastT2.tex}
    %     \caption{Die T2 Signale bei der jeweiligen Konzentration}
    % \end{figure}
    %  ich glaube die Abbildung ist nicht gut genug, 
    % bzw. die Fehlerdiskussion hätte ich keine Ahnung, warum 1000mol in der Mitte von den zwei ist

