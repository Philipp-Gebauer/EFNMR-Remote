% !TEX root = main.tex
\subsection{\textcolor{red}{ToDo!}Fourietrafo der Messungen mit unterschieldicheer Polarisationszeit}
\label{sec:Signalintensitaet}
Da der zweite Teil des Eperiments an einem anderen Versuchstag durchgeführt wurde, wurde zu Beginn erneut eine Optimierung des Aufbaus auf die Wasserprobe vorgenommen. Dies geschah analog zum Vorgehen, welches in Abschnitt \ref{sec:OptimizationandCharacterisationofFIDinwatersample} dargelegt wurde.
\begin{figure}[H]
    \centering
    % GNUPLOT: LaTeX picture with Postscript
\begingroup
  % Encoding inside the plot.  In the header of your document, this encoding
  % should to defined, e.g., by using
  % \usepackage[cp1252,<other encodings>]{inputenc}
  \inputencoding{cp1252}%
  \makeatletter
  \providecommand\color[2][]{%
    \GenericError{(gnuplot) \space\space\space\@spaces}{%
      Package color not loaded in conjunction with
      terminal option `colourtext'%
    }{See the gnuplot documentation for explanation.%
    }{Either use 'blacktext' in gnuplot or load the package
      color.sty in LaTeX.}%
    \renewcommand\color[2][]{}%
  }%
  \providecommand\includegraphics[2][]{%
    \GenericError{(gnuplot) \space\space\space\@spaces}{%
      Package graphicx or graphics not loaded%
    }{See the gnuplot documentation for explanation.%
    }{The gnuplot epslatex terminal needs graphicx.sty or graphics.sty.}%
    \renewcommand\includegraphics[2][]{}%
  }%
  \providecommand\rotatebox[2]{#2}%
  \@ifundefined{ifGPcolor}{%
    \newif\ifGPcolor
    \GPcolorfalse
  }{}%
  \@ifundefined{ifGPblacktext}{%
    \newif\ifGPblacktext
    \GPblacktexttrue
  }{}%
  % define a \g@addto@macro without @ in the name:
  \let\gplgaddtomacro\g@addto@macro
  % define empty templates for all commands taking text:
  \gdef\gplbacktext{}%
  \gdef\gplfronttext{}%
  \makeatother
  \ifGPblacktext
    % no textcolor at all
    \def\colorrgb#1{}%
    \def\colorgray#1{}%
  \else
    % gray or color?
    \ifGPcolor
      \def\colorrgb#1{\color[rgb]{#1}}%
      \def\colorgray#1{\color[gray]{#1}}%
      \expandafter\def\csname LTw\endcsname{\color{white}}%
      \expandafter\def\csname LTb\endcsname{\color{black}}%
      \expandafter\def\csname LTa\endcsname{\color{black}}%
      \expandafter\def\csname LT0\endcsname{\color[rgb]{1,0,0}}%
      \expandafter\def\csname LT1\endcsname{\color[rgb]{0,1,0}}%
      \expandafter\def\csname LT2\endcsname{\color[rgb]{0,0,1}}%
      \expandafter\def\csname LT3\endcsname{\color[rgb]{1,0,1}}%
      \expandafter\def\csname LT4\endcsname{\color[rgb]{0,1,1}}%
      \expandafter\def\csname LT5\endcsname{\color[rgb]{1,1,0}}%
      \expandafter\def\csname LT6\endcsname{\color[rgb]{0,0,0}}%
      \expandafter\def\csname LT7\endcsname{\color[rgb]{1,0.3,0}}%
      \expandafter\def\csname LT8\endcsname{\color[rgb]{0.5,0.5,0.5}}%
    \else
      % gray
      \def\colorrgb#1{\color{black}}%
      \def\colorgray#1{\color[gray]{#1}}%
      \expandafter\def\csname LTw\endcsname{\color{white}}%
      \expandafter\def\csname LTb\endcsname{\color{black}}%
      \expandafter\def\csname LTa\endcsname{\color{black}}%
      \expandafter\def\csname LT0\endcsname{\color{black}}%
      \expandafter\def\csname LT1\endcsname{\color{black}}%
      \expandafter\def\csname LT2\endcsname{\color{black}}%
      \expandafter\def\csname LT3\endcsname{\color{black}}%
      \expandafter\def\csname LT4\endcsname{\color{black}}%
      \expandafter\def\csname LT5\endcsname{\color{black}}%
      \expandafter\def\csname LT6\endcsname{\color{black}}%
      \expandafter\def\csname LT7\endcsname{\color{black}}%
      \expandafter\def\csname LT8\endcsname{\color{black}}%
    \fi
  \fi
    \setlength{\unitlength}{0.0500bp}%
    \ifx\gptboxheight\undefined%
      \newlength{\gptboxheight}%
      \newlength{\gptboxwidth}%
      \newsavebox{\gptboxtext}%
    \fi%
    \setlength{\fboxrule}{0.5pt}%
    \setlength{\fboxsep}{1pt}%
\begin{picture}(7200.00,5040.00)%
    \gplgaddtomacro\gplbacktext{%
      \csname LTb\endcsname%%
      \put(682,704){\makebox(0,0)[r]{\strut{}$0$}}%
      \put(682,1253){\makebox(0,0)[r]{\strut{}$10$}}%
      \put(682,1801){\makebox(0,0)[r]{\strut{}$20$}}%
      \put(682,2350){\makebox(0,0)[r]{\strut{}$30$}}%
      \put(682,2899){\makebox(0,0)[r]{\strut{}$40$}}%
      \put(682,3447){\makebox(0,0)[r]{\strut{}$50$}}%
      \put(682,3996){\makebox(0,0)[r]{\strut{}$60$}}%
      \put(682,4545){\makebox(0,0)[r]{\strut{}$70$}}%
      \put(814,484){\makebox(0,0){\strut{}$1820$}}%
      \put(1563,484){\makebox(0,0){\strut{}$1825$}}%
      \put(2311,484){\makebox(0,0){\strut{}$1830$}}%
      \put(3060,484){\makebox(0,0){\strut{}$1835$}}%
      \put(3809,484){\makebox(0,0){\strut{}$1840$}}%
      \put(4557,484){\makebox(0,0){\strut{}$1845$}}%
      \put(5306,484){\makebox(0,0){\strut{}$1850$}}%
      \put(6054,484){\makebox(0,0){\strut{}$1855$}}%
      \put(6803,484){\makebox(0,0){\strut{}$1860$}}%
    }%
    \gplgaddtomacro\gplfronttext{%
      \csname LTb\endcsname%%
      \put(308,2761){\rotatebox{-270}{\makebox(0,0){\strut{}Amplitude in willk\"urlicher Einheit}}}%
      \put(3808,154){\makebox(0,0){\strut{}Frequenz in $\si{\hertz}$}}%
      \csname LTb\endcsname%%
      \put(5858,4606){\makebox(0,0)[r]{\strut{}Polarisationszeit $\SI{4.0}{\second}$}}%
      \csname LTb\endcsname%%
      \put(5858,4386){\makebox(0,0)[r]{\strut{}Polarisationszeit $\SI{0.5}{\second}$}}%
    }%
    \gplbacktext
    \put(0,0){\includegraphics{plots/Polarisationszeit}}%
    \gplfronttext
  \end{picture}%
\endgroup

    \caption[Abhängigkeit der Signalintensität von der Polarisationszeit.]{In dieser Abbildung sind zwei ,,Pulse and Collect''-Messungen der bidestillierten Wasserprobe dargestellt. Die rote Kurve zeigt eine Messung mit einer Polarisationszeit von $\SI{4}{\second}$, bei der blauen Kurve beträgt die Polarisationszeit $\SI{0.5}{\second}$. Hierbei ist deutlich zu erkennen, dass die Signalintensität durch die höhere Polarisationszeit deutlich gesteigert werden kann. Bei den dargestellten Messungen um circa Faktor drei.} 
    \label{fig:SignalintensitaetPolarisationszeit}    
\end{figure}