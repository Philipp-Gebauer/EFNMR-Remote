% !TEX root = main.tex
\subsection{\textcolor{red}{ToDo!}Fourietrafo der Messungen mit unterschieldicheer Polarisationszeit}
\label{sec:Signalintensitaet}
Da der zweite Teil des Eperiments an einem anderen Versuchstag durchgeführt wurde, wurde zu Beginn erneut eine Optimierung des Aufbaus auf die Wasserprobe vorgenommen. Dies geschah analog zum Vorgehen, welches in Abschnitt \ref{sec:OptimizationandCharacterisationofFIDinwatersample} dargelegt wurde.
\begin{figure}[H]
    \centering
    % !TEX root = main.tex
\section{Fourietrafo der Messungen mit unterschieldicheer Polarisationszeit}
\begin{figure}[H]
    \centering
    % !TEX root = main.tex
\section{Fourietrafo der Messungen mit unterschieldicheer Polarisationszeit}
\begin{figure}[H]
    \centering
    % !TEX root = main.tex
\section{Fourietrafo der Messungen mit unterschieldicheer Polarisationszeit}
\begin{figure}[H]
    \centering
    \input{plots/Polarisationszeit.tex}
    \caption{Amplitude in abhängigkeit von zwei verschiedenen Piolarisatiosnzeiten}
\end{figure}
    \caption{Amplitude in abhängigkeit von zwei verschiedenen Piolarisatiosnzeiten}
\end{figure}
    \caption{Amplitude in abhängigkeit von zwei verschiedenen Piolarisatiosnzeiten}
\end{figure}
    \caption[Abhängigkeit der Signalintensität von der Polarisationszeit.]{In dieser Abbildung sind zwei ,,Pulse and Collect''-Messungen der bidestillierten Wasserprobe dargestellt. Die rote Kurve zeigt eine Messung mit einer Polarisationszeit von $\SI{4}{\second}$, bei der blauen Kurve beträgt die Polarisationszeit $\SI{0.5}{\second}$. Hierbei ist deutlich zu erkennen, dass die Signalintensität durch die höhere Polarisationszeit deutlich gesteigert werden kann. Bei den dargestellten Messungen um circa Faktor drei.} 
    \label{fig:SignalintensitaetPolarisationszeit}    
\end{figure}