% !TEX root = main.tex
\section{Grundlagen}
Die in diesem Abschnitt stichpunktartig erläuterten Themen sind bereits vorab mit dem Tutor in einem Vorkolloquium behandelt worden und sind nach Rücksprache hier nur noch kurz zu erwähnen. Bei genauerem Interesse wird hier auf die Literatur \cite{H1}, \cite{Usermanual} und \cite{AntennaResp} verwiesen:
\begin{itemize}
    \item Hyperfeinstruktur im Wasserstoffatom
    \begin{itemize}
        \item[→] Wechselwirkung von Kern- und Elektronenspin, Entstehung der 21-\si{\centi \metre}-Linie, deren Lebensdauer, Einfluss der \textsc{Doppler}-Verschiebung
    \end{itemize}
    \item Rotationsmodelle (differentiell, \textSC{Kepler} und starrer Körper) und deren Zusammenhang mit der Milchstraße und unserem Sonnensystem
    \item Koordinatensysteme (galaktisch, horizontal und äquatorial)
    \begin{itemize}
        \item[→] Wann sind welche Bereiche der Milchstraße am besten zu beobachten? Überprüfen der Beobachtungen via \textsc{Stellarium} 
    \end{itemize}
    \item Verständnis für die Größenordnungen
    \begin{itemize}
        \item[→] Milchstraße vs. Sonnensystem (Parsec [\si{\parsec}] und Lichtjahr [\si{\lightyear}])
    \end{itemize}
\end{itemize}