%----------------------------------------------------------------%
%--------------------------INFORMATIONEN-------------------------%
%----------------------------------------------------------------%
%	Infos gibt es zu jedem Paket auf www.ctan.org
%	Werden bei den Paketen bestimmte Optionen gesetzt, so sind die Wichtigsten erklaert
%	solange sie nicht selbsterklärend sind

%----------------------------------------------------------------%
%--------------------------GRUNDEINSTELLUNGEN--------------------%
%----------------------------------------------------------------%
%\documentclass[oneside, ngerman]{scrartcl}
\documentclass[oneside, english]{scrartcl}
%	'oneside'/'twoside': nicht zwischen linker und rechter Seite unterscheiden (alternativ twoside)
%	'twocolumn': wuerde 2 Spalten auf dem Blatt platzieren
%	'bibliography=totocnumbered': Normal nummeriertes Inhaltsverzeichnis (Kapitelnummer)
%	'listof=totocnumbered': Abbildungs- und Tabellenverzeichnis normal nummeriert (Kapitelnummer)
%	'ngerman' verwendet deutsch als Dokumentensprache (z.B. fuer Sirange)

\usepackage[english,ngerman]{babel}							%	Einstellen der Sprache
\usepackage[T1]{fontenc}							%	Wie wird Text ausgegeben, d.h. im PDF
\usepackage[utf8]{inputenc}							%	Welche Zeichen 'versteht' LaTeX bei der Eingabe?
\usepackage{lmodern}								%	Laedt Schriften, die geglaettet sind
											
\usepackage{blindtext}								%	Beispieltext, zum Testen geeignet
\usepackage{subfigure}
%----------------------------------------------------------------%
%--------------------------SEITENLAYOUT--------------------------%
%----------------------------------------------------------------%
\usepackage[left=3cm,right=3cm]{geometry}			%	Paket, welches vielfaeltige Einstellungen zum Seitenlayout liefert

%----------------------------------------------------------------%
%--------------------------ABSTÄNDE------------------------------%
%----------------------------------------------------------------%
\usepackage[onehalfspacing]{setspace}				%	Für Zeilenabstaende: 'singlespacing' (einfach), 'onehalfspacing' (1.5-fach), 'doublespacing' (2fach)

%\setlength{\parindent}{0cm}						%	Laengenangabe für die Einrueckung der ersten Zeile eines neuen Absatzes.
%\setlength{\parskip}{6pt plus 3pt minus 3pt}		%	Laengenangabe für den Abstand zwischen zwei Absaetzen.
%	Wenn diese beiden Befehle nicht kommentiert sind, wird ein Absatz nicht eingezogen sondern es gibt einen Abstand

%----------------------------------------------------------------%
%--------------------------MATHE---------------------------------%
%----------------------------------------------------------------%
\usepackage[]{mathtools}							%	Erweiterung von AMSMath, laedt automatisch AMSMath - für viele Mathe-Werkzeuge, 'fleqn' als Option ist für Mathe linksbuendig
\usepackage{amsfonts}								%	Für eine Vielzahl an mathematischen Symbolen

%----------------------------------------------------------------%
%--------------------------KOPF- UND FUSSZEILEN------------------%
%----------------------------------------------------------------%
\usepackage[automark,headsepline=.4pt]{scrlayer-scrpage}
\pagestyle{scrheadings}
\setkomafont{pageheadfoot}{\normalfont\bfseries}	%	Normale Schriftart und Fett für den Seitenkopf
\addtokomafont{pagenumber}{\normalfont\bfseries}	%	Normale Schriftart und Fett für die Seitenzahl
\clearscrheadfoot
\rohead{\thepage}									%	Rechter Seitenkopf mit Seitenzahl, ungerade Seiten
\lehead{\thepage}									%	Linker Seitenkopf mit Seitenzahl, gerade Seiten
\lohead{\headmark}									%	Linker Seitenkopf mit section, ungerade Seiten
\rehead{\headmark}									%	Linker Seitenkopf mit section, gerade Seiten
\lefoot[\pagemark]{\empty}							%	Leere Fußzeile, ungerade Seiten
\rofoot[\pagemark]{\empty}							%	Leere Fußzeile, gerade Seiten
\setlength{\headheight}{1.1\baselineskip}			%	Hoehe der Kopfzeile definieren
%	Definert man oben in der documentclass 'twoside', so wird zwischen geraden und ungeraden Seiten unterschieden

%----------------------------------------------------------------%
%--------------------------BILDER--------------------------------%
%----------------------------------------------------------------%
\usepackage{graphicx}									%	Um Bilder einbinden zu koennen
\usepackage[usenames,dvipsnames,svgnames]{xcolor}		%	Um Farben verwenden zu koennen
\usepackage{pdfpages}									%	pdfs importieren

%----------------------------------------------------------------%
%--------------------------POSITIONIERUNG------------------------%
%----------------------------------------------------------------%
\usepackage{float}

%----------------------------------------------------------------%
%--------------------------LISTEN--------------------------------%
%----------------------------------------------------------------%
\usepackage{enumitem}							%	Um Listen / Aufzaehlungen leichter zu modifizieren
%\setlist{noitemsep}							%	Verringert den Abstand in Aufzaehlungen

%----------------------------------------------------------------%
%--------TABELLEN-/BILDUNTERSCHRIFTEN und NUMMERIERUNG-----------%
%----------------------------------------------------------------%
\usepackage[format=hang, indention=0mm, labelsep=colon, justification=justified,  labelfont=bf]{caption}
\setlength\parindent{0pt}
%	'format=hang': Platz unter Abb. X bleibt frei, 'format=plain': auch unter Abb. X befindet sich Text
%	'idention': Einzug der zweiten Textzeile
%	'labelsep=colon': Trenner zwischen Nr. und Text ist Doppelpunkt und Leerzeichen
%	'justification=justified': Text wird als Block gesetzt
%	'labelfont=bf': 'Abbildung X.X' wird fett geschrieben

\numberwithin{equation}{section}				%	Nummerierung der Gleichungen, Tabellen und Bilder nach der Kapitelnummer
\numberwithin{figure}{section}
\numberwithin{table}{section}

%----------------------------------------------------------------%
%--------------------------LITERATURVERZEICHNIS------------------%
%----------------------------------------------------------------%
\usepackage[german]{babelbib}					%	Bereitstellung des deutschen Layouts fuer die Bibliography

%----------------------------------------------------------------%
%--------------------------SIUNITX-------------------------------%
%----------------------------------------------------------------%
\usepackage[]{siunitx}
\sisetup{locale = US}							%	Automatische Einstellung der Ausgabe für bestimmte Regionen (UK, US, DE, FR, ZA)

%----------------------------------------------------------------%
%--------------------------URLs / REFs---------------------------%
%----------------------------------------------------------------%
\usepackage[hidelinks]{hyperref}				%	Erweiterte Referenzierung ('hidelinks' verhindert Linien um Links)
\usepackage{gensymb}
\usepackage{subfigure}
\usepackage{ textcomp }
\usepackage{dsfont}
\usepackage{siunitx}
\usepackage{comment}
\usepackage{tikz}
\usepackage{epstopdf}
\usepackage{graphicx}
\usepackage{bookmark}
\usetikzlibrary{arrows.meta}
%----------------------------------------------------------------%
%--------------------------EIGENE BEFEHLE------------------------%
%----------------------------------------------------------------%
\DeclareSIUnit\parsec{pc}                               %   Neuer SI-Command Parsec
\DeclareSIUnit\lightyear{ly}                            %   Neuer SI-Command Lichtjahr
\newcommand{\textSC}[1]{{\normalfont \textsc{#1}}}      %   Für Kapitälchen in Überschriften
\DeclareMathOperator{\sinc}{sinc}                       %   sinc-Funktion
