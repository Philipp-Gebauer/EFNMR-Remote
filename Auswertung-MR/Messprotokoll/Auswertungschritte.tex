% !TEX root = Auswertungschritte.tex
\input{header.tex}
\usepackage{hyperref}
\usepackage{ wasysym }
\usepackage{MnSymbol,wasysym}
\begin{document}
\section{Noise measurement}
    \begin{itemize}
        \item Die Messdaten einfach auftragen
        \item erklären wie das Rauschen zustande kommen kann(ich vermute es kommt vom elektrischen Signal; weiß aber auch nicht was man sich darunter genau vorstellen kann, vllt Ausrichtung in Richtung des Erdmagnetfeldes oder auch Metallische gegenstände in der Umgebung vllt, durch die Messmethode im LCR Schwingkreis?)
        \item erklären, was ein gutes Rauschsignal ist $\SI[]{10}{\frac{\micro \volt}{\hertz}}$ und unter $\SI[]{3}{\frac{\micro \volt}{\hertz}}$ ist am besten; (Wir hatten ca. $\SI[]{7.5}{\frac{\micro \volt}{\hertz}}$)
    \end{itemize}

\section{Coil Analysis}
    Alles was unter optimization of FID gemacht wurde unter Punkt 1.)\\
    Kann glaub zusammengefasst werden weil es irgendwie keinen Sinn hat das getrennt zu betrachten\\
    Vllt wollen die hier explizit hören, dass für geringere Frequenzen die benötigte Kapazität für Resonanz steigt\\
    Vllt hier auf den LCR-Schwingkreis und die Messmethode genauer eingehen...

\section{Optimization of the FID in water sample}
\textbf{Alles was wir am Anfang gemacht haben um überhaupt messen zu können; sprich die Kapazität verändert, Autoshim eingestellt, $B_1$ duration und anregungsfrequenz (evtl. in welcher Reihenfolge man das gemacht hat und welche veränderungen das gegeben hat)}
    \textbf{1.}
    \begin{itemize}
        \item  Kurzen überblick geben, was eingestellt und was das Ziel ist. (So gut es geht einen scharfen Peak zu bekommen, indem das Signal so langsam wie möglich abflacht $\rightarrow$ durch Fourietrafo besseres Signal) Veränderung der Pulsdauer von $B_1$, fintuning am Homogenen Feld und LCR-Schwingkreis optimieren
        \item  Kapazität wurde gemessen; Analyse Coil wurde die Resonsnazfrequenz in abhängigkeit von der Kapazität aufgetragen. Reonanzfrequnz = soll als Lamorfrequenz gewählt werden (damit das Signal in der Umgebung der Lamorfrequenz von H stärker/deutlicher gemessen wird); die drei Spektren die man bekommen hat(eigentlich würde die Kurve mit der Kapazität reichen) 
        $\rightarrow$ daraus entnehmen wir die Kapazität für die Lamorfrequenz(wir haben $\SI[]{1837}{\hertz}$ genommen was einer Kapazität von ca. $\SI[]{13,8}{\nano \farad}$ entsprach)
    \end{itemize}

    \textbf{2.) Autoshimming}
    \begin{itemize}
        \item Ziel: Signal homogen wie möglich zu machen, iterationsverfahren des Computers aufzeichnen, welche Werte für den Versuch für die x,y,z-Achsen genommen wurden
    \end{itemize}

    \textbf{3.) $B_1$ dauer}
    \begin{itemize}
        \item das war die Messung mit dem komischen Buckel und dem scharfen Wasserstoffpeak(glaub wo man das Signal der Spule und das vom Wasserstoff gemessen hat. Timedelay 2ms oder so eingestellt hat um die beiden Signale zu erhalten)\\
        haben eigentlich 2 Messungen gemacht, wo das ganze noch nicht optimiert war. Da war der Peak nicht in der Mitte und das will man wohl nicht haben.
    \end{itemize}

\section{Characterisation of FID}
    Hier einen Plot von dem Wasserstoff Signal geben, welcher nach der Voreinstellung gemacht wurde. Hier dementsprechend die Theorie erklären was da passiert und was man sieht. Das gemessene Signal erläutern und wie man auf die Fourietrafo kommt.\\
    $\rightarrow$ Formel für die Fouriertrafo hinschreiben (vllt auch ein paar Beispiele für Fourietrafos machen)

\section{Longitudinal relaxation measurements}
    Theorie von $T_1$ erklären; was ist der unterschied im Erdmagnetfeld/Polarisationsfeld\\
    \textbf{$T_1,B_E$ und $T_1,B_P$ vermessung}\\
    Hierzu einfach die txt Datei plotten. Das sind jeweils die Datenpunkte die gemacht wurden\\
    FÜr $T_1B_E$ muss man das ganze mit der folgenden Formel plotten:
    \begin{align}
        S(t)= S_p exp(-t/T_1)
    \end{align}
    das ist das Signal was man ja misst. Die frage ist ob man das dann in der Magnetisierung dann angbit.\\
    Altprotokolle haben das ganze Wohl normiert $\frac{E}{E_0}$ (Vermute haben das ganze dann einfach durch den Anfangswert geteilt jeweils)\\
    \newline

    FÜr das Polarisierungsfeld wird das ganze mit der Formel 
    \begin{align}
        S(\tau_p)=S_p[1-exp(-\tau_p/T_1)]
    \end{align}
    genau das gleiche wie bei dem Erdmagnetfeld muss man jedoch beachten

\section{Hahn echo}
    \textbf{hier soll man sagen, welchen einfluss das shimming auf das Spin-Hahn echo hat;\\
    also welchen Einfluss das inhomogene Feld auf die Pulse und das flippen der Spins hat}\\
    -Grundlage von dem 90 und 180 grad Pulsen erklären\\
    -ganz wichtig, warum man das Feld inhomogen gemacht hat, vllt auch warum nur die x-Achse verändert wurde\\
    (ich glaube liegt daran, weil man nur in der x-Achse detektiert und man nur in dieser Achse 2 Spulen ansteuern kann)\\
    -wo kann man $T_2$ und $T_2^{\ast}$ messen\\
    -die Veränderungen der shimmingwerte erläutern und erklären warum 0 shimming bei x-gut ist
    \\
    -das gleiche wie beim FID machen, also Signal und die Fourietrafo b etrachten\\
    $\rightarrow$ das ganze  it dem FID vergleichen und welchen unterschied das Shimming macht...

\section{Multiple echo sequences}
    -erklären was man durch die aufeinanderfolgenden Pulse erreichen will\\
    - ich glaube wir haben zum einen noch verändert, indem man das ganze anders Pulst/oder andere Phase\\
    (habe ich nicht ganz gerafft, war das mit den 90 die man dann verändert hat)
    - wir haben unseren 180-Grad länge verändert, weil unser Signal zu gut war :) \smiley{}\\
    -in der Anleitung steht so etwas wie examin phases...
\section{Transversal relaxation measurements}
    \textbf{$T_2$ und $T_2^{\ast}$}
    \begin{align}
        E(\tau_e)=E_0exp(-2\tau_e/T_2)
    \end{align}
    Wir haben das zum einen $\tau_e$ verändert damit der PC irgendwie aus der stärke des Signals berechnen kann, bei welchem $\tau_e$ die Spins am meisten in Phase sind\\
    Außerdem wurde für $T_2^{\ast}$ das shimming verändert, um das zu vermessen oder so.
    \begin{align}
        \frac{1}{T_2^{\ast}}=\frac{1}{T_2}+\gamma \Delta B_0
    \end{align}
    keine Ahnung ob man die Foprmel braucht. Hab bei dem Teil sowiso noch nicht durchgeblickt...



\end{document}