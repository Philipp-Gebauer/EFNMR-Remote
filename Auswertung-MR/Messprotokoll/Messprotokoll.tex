% !TEX root = Messprotokoll.tex
%----------------------------------------------------------------%
%--------------------------INFORMATIONEN-------------------------%
%----------------------------------------------------------------%
%	Infos gibt es zu jedem Paket auf www.ctan.org
%	Werden bei den Paketen bestimmte Optionen gesetzt, so sind die Wichtigsten erklaert
%	solange sie nicht selbsterklärend sind

%----------------------------------------------------------------%
%--------------------------GRUNDEINSTELLUNGEN--------------------%
%----------------------------------------------------------------%
%\documentclass[oneside, ngerman]{scrartcl}
\documentclass[oneside, english]{scrartcl}
%	'oneside'/'twoside': nicht zwischen linker und rechter Seite unterscheiden (alternativ twoside)
%	'twocolumn': wuerde 2 Spalten auf dem Blatt platzieren
%	'bibliography=totocnumbered': Normal nummeriertes Inhaltsverzeichnis (Kapitelnummer)
%	'listof=totocnumbered': Abbildungs- und Tabellenverzeichnis normal nummeriert (Kapitelnummer)
%	'ngerman' verwendet deutsch als Dokumentensprache (z.B. fuer Sirange)

\usepackage[english,ngerman]{babel}							%	Einstellen der Sprache
\usepackage[T1]{fontenc}							%	Wie wird Text ausgegeben, d.h. im PDF
\usepackage[utf8]{inputenc}							%	Welche Zeichen 'versteht' LaTeX bei der Eingabe?
\usepackage{lmodern}								%	Laedt Schriften, die geglaettet sind
											
\usepackage{blindtext}								%	Beispieltext, zum Testen geeignet
\usepackage{subfigure}
%----------------------------------------------------------------%
%--------------------------SEITENLAYOUT--------------------------%
%----------------------------------------------------------------%
\usepackage[left=3cm,right=3cm]{geometry}			%	Paket, welches vielfaeltige Einstellungen zum Seitenlayout liefert

%----------------------------------------------------------------%
%--------------------------ABSTÄNDE------------------------------%
%----------------------------------------------------------------%
\usepackage[onehalfspacing]{setspace}				%	Für Zeilenabstaende: 'singlespacing' (einfach), 'onehalfspacing' (1.5-fach), 'doublespacing' (2fach)

%\setlength{\parindent}{0cm}						%	Laengenangabe für die Einrueckung der ersten Zeile eines neuen Absatzes.
%\setlength{\parskip}{6pt plus 3pt minus 3pt}		%	Laengenangabe für den Abstand zwischen zwei Absaetzen.
%	Wenn diese beiden Befehle nicht kommentiert sind, wird ein Absatz nicht eingezogen sondern es gibt einen Abstand

%----------------------------------------------------------------%
%--------------------------MATHE---------------------------------%
%----------------------------------------------------------------%
\usepackage[]{mathtools}							%	Erweiterung von AMSMath, laedt automatisch AMSMath - für viele Mathe-Werkzeuge, 'fleqn' als Option ist für Mathe linksbuendig
\usepackage{amsfonts}								%	Für eine Vielzahl an mathematischen Symbolen

%----------------------------------------------------------------%
%--------------------------KOPF- UND FUSSZEILEN------------------%
%----------------------------------------------------------------%
\usepackage[automark,headsepline=.4pt]{scrlayer-scrpage}
\pagestyle{scrheadings}
\setkomafont{pageheadfoot}{\normalfont\bfseries}	%	Normale Schriftart und Fett für den Seitenkopf
\addtokomafont{pagenumber}{\normalfont\bfseries}	%	Normale Schriftart und Fett für die Seitenzahl
\clearscrheadfoot
\rohead{\thepage}									%	Rechter Seitenkopf mit Seitenzahl, ungerade Seiten
\lehead{\thepage}									%	Linker Seitenkopf mit Seitenzahl, gerade Seiten
\lohead{\headmark}									%	Linker Seitenkopf mit section, ungerade Seiten
\rehead{\headmark}									%	Linker Seitenkopf mit section, gerade Seiten
\lefoot[\pagemark]{\empty}							%	Leere Fußzeile, ungerade Seiten
\rofoot[\pagemark]{\empty}							%	Leere Fußzeile, gerade Seiten
\setlength{\headheight}{1.1\baselineskip}			%	Hoehe der Kopfzeile definieren
%	Definert man oben in der documentclass 'twoside', so wird zwischen geraden und ungeraden Seiten unterschieden

%----------------------------------------------------------------%
%--------------------------BILDER--------------------------------%
%----------------------------------------------------------------%
\usepackage{graphicx}									%	Um Bilder einbinden zu koennen
\usepackage[usenames,dvipsnames,svgnames]{xcolor}		%	Um Farben verwenden zu koennen
\usepackage{pdfpages}									%	pdfs importieren

%----------------------------------------------------------------%
%--------------------------POSITIONIERUNG------------------------%
%----------------------------------------------------------------%
\usepackage{float}

%----------------------------------------------------------------%
%--------------------------LISTEN--------------------------------%
%----------------------------------------------------------------%
\usepackage{enumitem}							%	Um Listen / Aufzaehlungen leichter zu modifizieren
%\setlist{noitemsep}							%	Verringert den Abstand in Aufzaehlungen

%----------------------------------------------------------------%
%--------TABELLEN-/BILDUNTERSCHRIFTEN und NUMMERIERUNG-----------%
%----------------------------------------------------------------%
\usepackage[format=hang, indention=0mm, labelsep=colon, justification=justified,  labelfont=bf]{caption}
\setlength\parindent{0pt}
%	'format=hang': Platz unter Abb. X bleibt frei, 'format=plain': auch unter Abb. X befindet sich Text
%	'idention': Einzug der zweiten Textzeile
%	'labelsep=colon': Trenner zwischen Nr. und Text ist Doppelpunkt und Leerzeichen
%	'justification=justified': Text wird als Block gesetzt
%	'labelfont=bf': 'Abbildung X.X' wird fett geschrieben

\numberwithin{equation}{section}				%	Nummerierung der Gleichungen, Tabellen und Bilder nach der Kapitelnummer
\numberwithin{figure}{section}
\numberwithin{table}{section}

%----------------------------------------------------------------%
%--------------------------LITERATURVERZEICHNIS------------------%
%----------------------------------------------------------------%
\usepackage[german]{babelbib}					%	Bereitstellung des deutschen Layouts fuer die Bibliography

%----------------------------------------------------------------%
%--------------------------SIUNITX-------------------------------%
%----------------------------------------------------------------%
\usepackage[]{siunitx}
\sisetup{locale = DE}							%	Automatische Einstellung der Ausgabe für bestimmte Regionen (UK, US, DE, FR, ZA)

%----------------------------------------------------------------%
%--------------------------URLs / REFs---------------------------%
%----------------------------------------------------------------%
\usepackage[hidelinks]{hyperref}				%	Erweiterte Referenzierung ('hidelinks' verhindert Linien um Links)
\usepackage{gensymb}
\usepackage{subfigure}
\usepackage{ textcomp }
\usepackage{dsfont}
\usepackage{siunitx}
\usepackage{comment}
\usepackage{tikz}
\usepackage{epstopdf}
\usepackage{graphicx}
\usetikzlibrary{arrows.meta}
%----------------------------------------------------------------%
%--------------------------EIGENE BEFEHLE------------------------%
%----------------------------------------------------------------%
\DeclareSIUnit\parsec{pc}                               %   Neuer SI-Command Parsec
\DeclareSIUnit\lightyear{ly}                            %   Neuer SI-Command Lichtjahr
\newcommand{\textSC}[1]{{\normalfont \textsc{#1}}}      %   Für Kapitälchen in Überschriften
\DeclareMathOperator{\sinc}{sinc}                       %   sinc-Funktion

\usepackage{hyperref}
\usepackage{ wasysym }
\begin{document}
\selectlanguage{english}
\textbf{Measureplan for the experiment EFNMR 1 Remote}\\
Experiment planned on July 9th 2020 at 18 o'clock\\
executor: Marc Neumann and Philipp Gebauer and Simon Keegan
    
    \begin{enumerate}
        \item login \& get used to the program; \\
        \textbf{making the data file with the names} (until 18:30) $\square$
        \item everything is connected and aligned in the earth magnetic Field \checked
        \item EFNMR $\longrightarrow$ \textbf{MonitorNoise} (dependent of the location and the orientation of the probe); ask tutor if the program changes C automatically; $gamma_{H1} = \SI{42.577}{\frac{MHz}{T}}\cdot 2 \pi$\\
        noise level less than $\SI{10}{\frac{\micro\volt}{\hertz}}$  is okay; less than $\SI{5}{\frac{\micro\volt}{\hertz}}$ is good and fewer than $\SI{3}{\frac{\micro\volt}{\hertz}}$ is great (about $\SI{30}{\min}$) $\square$ 
        \item  To investigate the $B_1$ transmit/receive coil; 
        EFNMR $\longrightarrow$ AnalyseCoil $\longrightarrow$ click Analyse (note the values from the CLI);\\
        \textbf{measure the resonance frequency dependent from the Capacity}; \\
        In the figure of the script from $\SI{0}{\nano\farad}$ to $\SI{20}{\nano\farad}$ in $\SI{1/2}{\nano\farad}$ steps? (about $\SI{60}{min}$)
        \item \textbf{detect the hydrogen-signal in the water probe}\\ 
        EFNMR $\longrightarrow$ PulsAndCollect $\longrightarrow$ measure the spektrum and change the values from $B_1, C$ and the \glqq shimming\grqq{} to get an better Signal; $B_1$ minimum and step size should be the half of the Lamor frequence  ($\SI{60}{min}$) \textbf{every change and step should be noted, so you can reproduce the  simulation every time}
        \item \textbf{measure the longitudinal spin relaxation in the polarised magnetic field and in the earth magnetic field}\\
        measure once $\tau_p$ in the polarized field and then $t$ the time between the polarisation and the pulse in the magnetic field\\  (repeat a few times to get a good signal; $\SI{60}{\min}$)$\square$
        \item \textbf{measure the amplitude and the integral of the spectral peak by different shimming}; measure $T_2,T_2^{\ast}$\\
        ($\SI{30}{min}$) $\square$
        \item \textbf{measure the puls from many spin-echos}\\
        try to change the the time between the pulses($\SI{45}{\min}$) $\square$
        
        
    \end{enumerate}
    \newpage
    \selectlanguage{ngerman}
    \begin{enumerate}
        \item Einloggen und einrichten des Dateipfades (18:20)
        \item EFNMR Werte von dem ersten Versuchstag wiederholen oder als gelich angenommen werden(Teil 6.4.1 im Usermanual)(Am ersten Versuchstag ca. 3 Stunden eingeplant)
        \item $T_1$ und $T_2$ werden jeweils in Abhängigkeit von der Konzentration der Lösung bestimmt(Welche Auswirkung das $CUSO_4$ (paramagnetisches Salz?) auf das $H^1$ Spektrum vom Wasser hat; Proben enthalten  Konzentratrionen zwischen $\SI{250}{\mu \textsc{M}}$ bis $\SI{5000}{\mu \textsc{M}}$ $\Longrightarrow$ Auftragen der Relaxationszeiten über die Konzentration um Relaxitivität $r_1$ bzw. $r_2$ erhalten; (2 Stunden)
        \item 1-D Bild erstellen mit Gradient echo imaging;(30 min)
        \item Untersuchung der Kopplungskonstante (J-Kopplung) von 2,2,2 Flurethanol; Signalaufnahme mit veränderten anregungsfrequenzen und veränderter Resonanzbedingung im Schwingkreis (1 Stunde)
        \item 2-D nD Gradient-Echo; veränderung des Gradientens entlang der x,y-Achse und zusätzlich die Gewichtung der Relaxationszeit, sodass man die Röhren einzelnd sehen kann (30 min)
        \item PGSE-Experiment; nach dem $90^{\degree}$ und $180^{\degree}$ Puls werden jeweils kurze Gradientenfelder angelegt um auf die Selbstdiffusion im Wasser zu untersuchen (30 min)
        \end{enumerate}
\end{document}